\documentclass{article}
%%%%%%%%%%%%%%%%%%%%%%%%%%%%%%%%%%%%%%%%%%%%%%%%%%%%%%%%%%%%%%%%%%%%%%%%%%%%%%%%%%%%%%%%%%%%%%%%%%%%%%%%%%%%%%%%%%%%%%%%%%%%%%%%%%%%%%%%%%%%%%%%%%%%%%%%%%%%%%%%%%%%%%%%%%%%%%%%%%%%%%%%%%%%%%%%%%%%%%%%%%%%%%%%%%%%%%%%%%%%%%%%%%%%%%%%%%%%%%%%%%%%%%%%%%%%
\usepackage{amsmath,amssymb}
% more math
\usepackage{amsfonts}
\usepackage{amssymb}
\usepackage{amstext}
\usepackage{amsbsy}

\usepackage{color}
\newcommand{\mt}[1]{\marginpar{\small #1}}
%%%%%%%%%%%%%%%%%%%%%%%%%%%%%%%%%%%%%%%%%%%%%%%%%%%%%%%%%%%%%%%%%%%%
% new commands
\newcommand{\nc}{\newcommand}
% operators
\renewcommand{\div}{\vec{\nabla}\! \cdot \!}
\newcommand{\grad}{\vec{\nabla}}
% latex shortcuts
\newcommand{\bea}{\begin{eqnarray}}
\newcommand{\eea}{\end{eqnarray}}
\newcommand{\be}{\begin{equation}}
\newcommand{\ee}{\end{equation}}
\newcommand{\bal}{\begin{align}}
\newcommand{\eali}{\end{align}}
\newcommand{\bi}{\begin{itemize}}
\newcommand{\ei}{\end{itemize}}
\newcommand{\ben}{\begin{enumerate}}
\newcommand{\een}{\end{enumerate}}
% DGFEM commands
\newcommand{\jmp}[1]{[\![#1]\!]}                     % jump
\newcommand{\mvl}[1]{\{\!\!\{#1\}\!\!\}}             % mean value
\newcommand{\keff}{\ensuremath{k_{\textit{eff}}}\xspace}
% shortcut for domain notation
\newcommand{\D}{\mathcal{D}}
% vector shortcuts
\newcommand{\vo}{\vec{\Omega}}
\newcommand{\vr}{\vec{r}}
\newcommand{\vn}{\vec{n}}
\newcommand{\vnk}{\vec{\mathbf{n}}}
\newcommand{\vj}{\vec{J}}
% extra space
\newcommand{\qq}{\quad\quad}
% common reference commands
\newcommand{\eqt}[1]{Eq.~(\ref{#1})}                     % equation
\newcommand{\fig}[1]{Fig.~\ref{#1}}                      % figure
\newcommand{\tbl}[1]{Table~\ref{#1}}                     % table

\newcommand{\ud}{\,\mathrm{d}}

\newcommand{\tcr}[1]{\textcolor{red}{#1}}
%%%%%%%%%%%%%%%%%%%%%%%%%%%%%%%%%%%%%%%%%%%%%%%%%%%%%%%%%%%%%%%%%%%%

\begin{document}

\begin{center}
{ \Large Answers to Reviewer \#1}
\end{center}

\bigskip

\noindent Ms. Ref. No. JCOMP-D-14-00443\\
Old Title: ``Numerical solution of the 1-D Grey non-equilibrium Radiation-Hydrodynamic equations with an entropy-based artificial viscosity technique'', \\
New Title: ``Entropy-based artificial viscosity stabilization for non-equilibrium Grey Radiation-Hydrodynamics'',\\
{\it Journal of Computational Physics}\\

\bigskip
\bigskip

{
\color{blue}
The authors extend the entropy-based artificial viscosity method (Refs. 2,3) to non-equilibrium diffusion, radiation hydrodynamics. This is certainly novel research. However, I have some very serious issues with the theory development and its applicability. I also don't believe that the results demonstrate, for this particular radiation model, the need for any more regularization than would for hydro alone when the shock transition layer is resolved, or is the correct regularization when it's unresolved. Therefore, before considering publication, I believe this paper must address these concerns.
My feeling is that, unfortunately, the changes I require are such that the authors may consider withdrawing this paper. I do hope they continue to pursue the entropy viscosity approach, as I believe it has merit. There's also nothing wrong with the numerical solution technique chosen. But for rad-hydro, the arguments for the particular entropy viscosity used has serious issues, and I believe will lead to confusion in the community. The authors may come up with a way around my concerns here that I have not thought of, or I could simply be wrong. If so, I look forward to their response.
The details of my comments and concerns are given below.\\

1. After eq. (1), the constant a is more commonly referred to as the radiation constant. It's related to the Stefan-Boltzmann (or Stefan) constant by the relation a = 4/c. The Boltzmann constant is usually written as k or k$_B$, such as the kT term in the Planck function. Note that there's also a Boltzmann number used in the rad-hydro literature, which is different from all of these.
}

This has been corrected. Thank you.
\bigskip


{
\color{blue}
2. For references with multiple authors, only the initials for some of the authors are given (including for the surnames) and in the wrong order. As example, the authors of Ref. 4 should be Dai, W. W. and Woodard, P. R. but in this paper are given as W. P. R. Dai W.}

This has been corrected.  Thank you.
\bigskip


{
\color{blue}
3. It should be emphasized that the radiation model is grey non-equilibrium diffusion. Non-equilibrium diffusion (NED) should definitely should be mentioned in the abstract, and I would strongly suggest also in the title. As my comments below will hopefully make clear, the approach used in this study may at best be restricted to non-equilibrium diffusion. Specifically, it's not at all clear whether the approach will work for full transport (such as discrete ordinates), or for that matter, whether difficulties will be encountered in the equilibrium-diffusion limit, even though the NED equations reduce to the equilibrium-diffusion rad- hydro equations in the optically-thick limit. More details below.
}

We have modified the title and the manuscript to reflect this (grey non-equilibrium diffusion is now stated wherever needed).
\bigskip


{
\color{blue}
The main issue I have with this paper is using the system (2) as the hyperbolic system to develop an entropy viscosity. In particular:
}
\bigskip


{
\color{blue}
4. I disagree that the system (2) is the frozen approximation for NED. It cannot be found through some limiting value of the opacity. The authors state that (2) is found in the limit of infinite opacity. This is incorrect. In this limit, the relaxation term dominates, leading to radiative equilibrium and the strong equilibrium limit (equilibrium- diffusion rad-hydro without diffusion; Lowrie, Morel, Hittinger ApJ 1999). This is not the system (2).
}

The reviewer is correct: the system (2) is not the frozen approximation and should not be referred to as such. We have corrected this. 
\bigskip


{
\color{blue}
5. Moreover, if (2) applied, then in Ref [11] sect. 4.2, the shock-jump relations would have to include the nonconservative product $u \partial_x \epsilon$. But
I believe Ref. [11] is correct, in that this product does not contribute to the shock-jump relation, because $\epsilon$ is continuous through the (hydro) shock as a result of the regularization of the radiation diffusion. But this result is not true of the system (2), which has dropped the diffusion term. As discussed in Ref. [11], shocks that are embedded in the shock profile satisfy the unmodified hydro-jump relations. This is also true for rad-hydro using full transport. At least for the Eulerian frame, the frozen problem decouples radiation and hydro. In that case, it makes sense to develop entropy-based viscosities separately for the radiation and hydro (and the hydro-alone case has already been done in the References). If the authors feel that the system (2) is appropriate, they need a much stronger argument that it is.
\\}
We have emphasized that several techniques are based on the analysis of the characteristics of the hyperbolic portion which justifies the use of system (2) as a starting point. System (2) was originally used to derive the viscous regularization (functional form of the dissipative terms and proof of the entropy minimum principle) and also a functional form for the entropy function for the non-equilibrium GRH equations.\\

However, the reviewer's remark made us dig deeper into this question. Following the same steps shown in Appendix A, we can demonstrate that, with the same entropy function, the entropy minimum principle holds when including the diffusion relaxation source terms that appear on the RHS of the full GRH equations (not just their hyperbolic parts). We believe this to be an very important new result and it should answer the concerns expressed in the reviewer's remark. 

\bigskip


{
\color{blue}
6. On a related note, if system (2) is the frozen system, then shocks em- bedded in the shock transition layer should satisfy its jump relations, whatever those may be. Deriving the jump relations for a nonconservative system is not straightforward. This would also imply that the jump relations in Ref. [11], sect 4.2 are wrong, and so its overall solutions are wrong. The authors need to make that case, if they believe that's true.
}

See \# 4.
\bigskip


{
\color{blue}
7. As further evidence of the inappropriateness of the system (2), consider its sound speed, given by eq. (6). The importance of the sound speed is that typically artificial viscosities scale with some power of the sound speed (such in eq. (9)), so that a bad choice can mean either too much or too little regularization. A dispersion analysis of rad-hydro using P1 (Lowrie, Morel, \& Hittinger, ApJ 1999), which certainly is closely related to NED, shows 3 regimes based on their sound speeds: the unmodified material sound speed (frozen for P1, full transport, and I believe also NED), isothermal sound speed, and the equilibrium- diffusion modified sound speed (see sect. 4.1 of Ref 10). Generally, none of these sound speeds match eq. (6), except for the equilibrium- diffusion speed for the very special case of a $\gamma$-law gas with $\gamma = 4/3$. So it's not clear that the sound speed (6) will result in a sufficient viscosity in the equilibrium diffusion limit, and may be too much in the streaming limit. Again, these facts also call into question the appropriateness of basing an entropy and entropy viscosity on the hyperbolic system (2).
}

We added a section titled ``Scaling of the entropy function derived for the non-equilibrium RHD equation in the Equilibrium-Diffusion Limit''  to show that the sound speed we use in the definition of the first-order viscosity coefficient corresponds to the maximum value of the sound speed (obtained for $\Gamma = 1/3$) derived in Ref. [11] and recalled in Eq.(24a).
\bigskip


{
\color{blue}
8. The system (2) cannot be written conservation form, as the authors state in the paragraph that follows. Mass (first equation), momentum (second), and total energy (add the last two equations) can be written in conservation form, but one of the last two equations must be retained to form a complete system. Nevertheless, it's hyperbolic (although the authors don't show this or need to). It's a nonconservative system, which might be fine, if this system was appropriate. I don't believe its true that conservation form ``allows us to assume the existence of an entropy function,'' as stated after eq. (5). Conservation form is not a requirement to define an entropy. But all this is a moot point, because I don't believe the system (2) is the appropriate hyperbolic system on which to develop an entropy.
}

The reviewer is correct when stating that the system (2) cannot be written in a conservation form. Nevertheless, the system (2) is proved to be hyperbolic in the paper since its eigenvalues, given in Eq.(2), are unconditionally real which allows us to assume the existence of an entropy function.\\

Furthermore, we have added a section on the equilibrium-diffusion limit of the regularized non-equilibrium GRH equations and we show that  
\begin{enumerate}
\item we recover the equilibrium-diffusion limit equations of Lowrie \& Morel (JQSR 2001; Ref. [13] in our revised manuscript) with dissipative terms
\item we recover the entropy function of Lowrie \& Morel (JQSR 2001; Ref. [13])
\item the regularized GRH equations possess the proper asymptotic limit.
\end{enumerate}
This further reinforces our opinion that the viscous regularization and the entropy function derived for the non-equil. GRH are the correct ones.\\

We have also shown that the following two operations commute:
\begin{enumerate}
\item regularize the non-equil. GRH and then perform an asymptotic limit
\item start with the equilibrium-diffusion limit equations and then add a viscous regularization
\end{enumerate}
 
\bigskip


{
\color{blue}
9. Sect. 4.1 of Ref. 10 developed an entropy function for the equilibrium diffusion limit. The authors don't address why the entropy they have selected instead is appropriate in this limit, even though the NED system reduces to equilibrium diffusion in the optically-thick limit.
}

In the section ``Scaling of the entropy function derived for the non-equilibrium RHD equation in the Equilibrium-Diffusion Limit'', an asymptotic limit is performed to demonstrate that the entropy we selected for the system (2) match the entropy $s^*$ derived in Ref. [13]\footnote{13 is the reference number in the revised version} in the equilibrium-diffusion limit. We believe this to be a strong argument for the validity of our approach.
\bigskip


{
\color{blue}
10. Finally, related to the last point, the radiative shock numerical solutions shown in this study all resolve the transition layer from the left to right states. Specifically, they resolve radiation diffusion term. When the transition layer is resolved, entropy viscosity is needed only to regularize the embedded hydro shocks that might occur, such as for the Mach 2 and 5 cases. But hydro shocks only need a viscosity appropriate for hydro-only shocks, developed in the entropy viscosity approach in earlier work. Of course, this is assuming the solutions of Ref. [11] are correct.
For the cases in this study, the left and right states are in radiative equilibrium. The jump relations are given in Ref [11] sect. 4.2 and the entropy in Sect. 4.1 of Ref. 10, which again, differs from the entropy in the current study. A numerical shock regularization is required when the transition layer is unresolved. Referring to Fig. 7, a spacing of 0.02 cm would not resolve the transition layer for this case. A spacing of 0.2 cm would be a challenge for the Mach 50 case (Fig. 14). The entropy viscosity would have to regularize the overall jumps in these cases, or else numerical oscillations would occur. Such cases must be explored, or else it's not clear whether the entropy viscosity is appropriate. It would also seem that the entropy developed in Sect. 5.1 of Ref. 10 would be more appropriate in such cases, and this should be addressed.
}

We added a test case in the section ``An equilibrium-diffusion test'' in the results section to discuss the test case described by the reviewer and showed that the viscous regularization efficiently stabilize the scheme when the viscous coefficient becomes larger than the diffusion coefficient. 
\bigskip

\end{document}