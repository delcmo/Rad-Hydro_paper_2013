% flddoc.tex V2.0, 13 May 2010

\documentclass[times]{fldauth}

\usepackage{moreverb}

%\usepackage[dvips,colorlinks,bookmarksopen,bookmarksnumbered,citecolor=red,urlcolor=red]{hyperref}
\usepackage[colorlinks,bookmarksopen,bookmarksnumbered,citecolor=red,urlcolor=red]{hyperref}

\newcommand\BibTeX{{\rmfamily B\kern-.05em \textsc{i\kern-.025em b}\kern-.08em
T\kern-.1667em\lower.7ex\hbox{E}\kern-.125emX}}

\def\volumeyear{2010}

%
%=================================================================================================
% new commands
% +++++++++++++++++++++++++++++++++++++++++++++++++++++++++++++++++++++++++++++++++++++++++++++++++
\newcommand{\nc}{\newcommand}

% operators
\renewcommand{\div}{\mbold{\nabla}\! \cdot \!}
\newcommand{\grad}{\mbold{\nabla}}
\newcommand{\divv}[1]{\boldsymbol{\nabla}^{#1}\! \cdot \!}
\newcommand{\gradd}[1]{\mbold{\nabla}^{#1}}
\newcommand{\mbold}[1]{\boldsymbol#1}
% latex shortcuts
\newcommand{\bea}{\begin{eqnarray}}
\newcommand{\eea}{\end{eqnarray}}
\newcommand{\be}{\begin{equation}}
\newcommand{\ee}{\end{equation}}
\newcommand{\bal}{\begin{align}}
\newcommand{\eali}{\end{align}}
\newcommand{\bi}{\begin{itemize}}
\newcommand{\ei}{\end{itemize}}
\newcommand{\ben}{\begin{enumerate}}
\newcommand{\een}{\end{enumerate}}
\usepackage{amsthm}
\newtheorem*{remark}{Remark}
% DGFEM commands
\newcommand{\jmp}[1]{[\![#1]\!]}                     % jump
\newcommand{\mvl}[1]{\{\!\!\{#1\}\!\!\}}             % mean value
\newcommand{\keff}{\ensuremath{k_{\textit{eff}}}\xspace}
% shortcut for domain notation
\newcommand{\D}{\mathcal{D}}
% vector shortcuts
\newcommand{\vo}{\mbold{\Omega}}
\newcommand{\vr}{\mbold{r}}
\newcommand{\vn}{\mbold{n}}
\newcommand{\vnk}{\mbold{\mathbf{n}}}
\newcommand{\vj}{\mbold{J}}
\newcommand{\eig}[1]{\| #1 \|_2}
%
\newcommand{\EI}{\mathcal{E}_h^i}
\newcommand{\ED}{\mathcal{E}_h^{\partial \D^d}}
\newcommand{\EN}{\mathcal{E}_h^{\partial \D^n}}
\newcommand{\ER}{\mathcal{E}_h^{\partial \D^r}}
\newcommand{\reg}{\textit{reg}}
%
\newcommand{\norm}{\textrm{norm}}
\renewcommand{\Re}{\textrm{Re}}
\newcommand{\Pe}{\textrm{P\'e}}
\renewcommand{\Pr}{\textrm{Pr}}
%
\newcommand{\resi}{R}
%\newcommand{\resinew}{\tilde{D}_e}
\newcommand{\resinew}{\widetilde{\resi}}
\newcommand{\resisource}{\widetilde{\resi}^{source}}
\newcommand{\matder}[1]{\frac{\textrm{D} #1}{\textrm{D} t}}
%
% extra space
\newcommand{\qq}{\quad\quad}
% common reference commands
\newcommand{\eqt}[1]{Eq.~(\ref{#1})}                     % equation
\newcommand{\fig}[1]{Fig.~\ref{#1}}                      % figure
\newcommand{\tbl}[1]{Table~\ref{#1}}                     % table
\newcommand{\sct}[1]{Section~\ref{#1}}                   % section
\newcommand{\app}[1]{Appendix~\ref{#1}}                   % appendix
%
\newcommand{\ie}{i.e.,\@\xspace}
\newcommand{\eg}{e.g.,\@\xspace}
\newcommand{\psc}[1]{{\sc {#1}}}
\newcommand{\rs}{\psc{R7}\xspace}
%
\newcommand\br{\mathbf{r}}
%\newcommand{\tf}{\varphi}
\newcommand{\tf}{b}
%
\newcommand{\tcr}[1]{\textcolor{red}{#1}}
\newcommand{\tcb}[1]{\textcolor{blue}{#1}}
\newcommand{\mt}[1]{\marginpar{ {\tiny #1}}}

\bibliographystyle{elsarticle-num}

\begin{document}

\runningheads{A.~N.~Other}{A demonstration of the \journalabb\
class file}

\title{A demonstration of the \LaTeXe\ class file for the
\itshape{\journalnamelc}\footnotemark[2]}

\author{A.~N.~Other\corrauth}

\address{John Wiley \& Sons, Ltd, The Atrium, Southern Gate, Chichester,
West Sussex, PO19~8SQ, UK}

\corraddr{Journals Production Department, John Wiley \& Sons, Ltd,
The Atrium, Southern Gate, Chichester, West Sussex, PO19~8SQ, UK.}

\begin{abstract}
This paper describes the use of the \LaTeXe\
\textsf{\journalclass} class file for setting papers for the
\emph{\journalnamelc}.
\end{abstract}

\keywords{class file; \LaTeXe; \emph{\journalabb}}

\maketitle

\footnotetext[2]{Please ensure that you use the most up to date
class file,
available from the FLD Home Page at\\
\href{http://www3.interscience.wiley.com/journal/2861/home}{\texttt{http://www3.interscience.wiley.com/journal/2861/home}}}

\vspace{-6pt}

\section{Introduction}
\vspace{-2pt}

Accurately solving for the non-equilibrium Grey diffusion Radiation-Hydrodynamic (GRH) equations (REF) with 
high-order accuracy has been an area of great interest and substantial research 
\tcr{if so, we should demonstrate this by providing a (long) list of references here \cite{Balsara,LowrieMorelHittinger,EdwardsMorelLowrie}}. 
The GRH equations form a non-conservative multi-physics system of equations and couples the inviscid 
hydrodynamic equations, i.e., Euler equations (REF)\tcr{do we need a ref here?}, with a 
radiation-diffusion equation. The coupling between the two physics is ensured through relaxation terms (REF) 
that contain neither spatial nor temporal derivative terms. When the material is optically thick, i.e., 
the absorption opacity becomes large, the GRH equations devolve to the Equilibrium-Diffusion Limit (EDL) 
equations (REF). The EDL can be obtained from the GRH by performing a Chapman-Enskog expansion (REF) on the
relaxation source terms.
% and can be written in a conservative form. 
Numerically solving for the GRH equations is challenging for at least three reasons: (i) because of the 
wave-dominated nature of the GRH equations, \emph{a numerical method} is required to preserve the 
solution's wave structure; 
(ii) the GRH equations cannot be written in a conservative form and thus, the notions of weak and entropy 
solutions cannot be defined in the sense of distributions (REF) \tcr{I would remove this: as is the case for 
conservative case (REFs)}. Instead, the theory developed by Dal Maso, Le Floch and Murat (DLM) must be used to
rigorously define, in a weak sense, the non-conservative products (REF) present in the material energy equation 
and radiation energy density equation;
(iii) finally, the \emph{numerical scheme} devised needs to preserve the equilibrium-diffusion limit in order to recover the 
EDL system of equations by ensuring that the viscous regularization dissipative terms are also
well-scaled in such a limit.

Substantial research efforts regarding the GRH equations have been devoted to the elaboration of 
numerical solution schemes, focusing mostly on approximate Riemann solvers for discontinuous schemes. In \cite{Balsara}, an 
approximate Riemann solver is developed for the GRH equations by considering the frozen approximation that 
uncouples the two physics, photon radiation and material hydrodynamics, but the validity of such an 
approximation may be questionable in the equilibrium-diffusion limit. Lowrie et al. \cite{LowrieMorelHittinger} 
proposed a generalized Riemann solver that accounts for the presence of relaxation terms. Edwards and al. 
\cite{EdwardsMorelLowrie} employed a two-stage semi-implicit IMEX scheme to solve the Radiation-Hydrodynamic 
equations where an approximate Riemann solver along with a flux limiter was used to resolve shocks and other waves. 
Their results show good agreement with semi-analytical solutions. \tcr{does not Edwards have 2 papers with Lowrie, 
one of the EDL limit as well? not useful here maybe}

The theory developed by DLM (REF) applies to non-conservative hyperbolic system of equations of the form given in \eqt{eq:nc-syst-eq}:
%
\begin{equation}\label{eq:nc-syst-eq}
\partial_t U + A(U) \partial_x U = 0 \, , U \in \Omega
\end{equation}
%
where $\Omega$ is an open convex set and $U$ is a vector solution. It is assumed that \eqt{eq:nc-syst-eq} cannot be 
written in a conservative form, i.e. the matrix $A(U)$ is \underline{not} the jacobian matrix of vector-values function $f: 
\mathbb{R}^n \to \mathbb{R}^n$ such as $A(U) = \frac{\partial f(U)}{\partial U}$. Systems of equations of given by  
\eqt{eq:nc-syst-eq} are often encountered in two-phase flow models (REFs), for instance. With the above assumptions, 
the non-conservative product $A(U) \partial_x U$ is not defined by the theory of distribution (REF) \tcr{in the sense of distribution?} when considering 
a Riemann problem, that is, one cannot rigorously define the notions of weak and entropy solutions as it is typically done for 
conservation laws (REFs). In the theory proposed by DLM, the non-conservative products are defined as a bounded Borel 
measure by introducing a Lipschitz family of path $\phi: [0,1] \times \mathbb{R}^n \times \mathbb{R}^n$ such that 
$\phi(0, U_L, U_R) = U_L$ and $\phi(1, U_L, U_R) = U_R$. Generally speaking, the main idea is to specify the values of the 
solution at points of discontinuity so that the non-conservative product admits a weak solution. The Borel measure of 
the non-conservative product is denoted by $\left[ A(U) \partial_x U \right]_\phi$ and is defined as
%
\begin{equation}
\left[ A(U) \partial_x U \right]_\phi = \int_B A(U) \partial_x U dx \, 
\end{equation}
%
if the solution $U$ is continuous, and as
%
\begin{equation}
\left[ A(U) \partial_x U \right]_\phi = \int_0^1 A(\phi(s,U_L,u_R)) \frac{\partial \phi}{\partial s}(s, U_L, U_R)ds 
\end{equation}
%
when the solution admits a discontinuity. The vector solutions $U_L$ and $U_R$ denote the left and the right limit values at the discontinuity. This formulation of the non-conservative product lets us introduce the notion 
of a Rankine-Hugoniot jump relation and generalizes the notion of shock, rarefaction, and contact waves 
for non-conservatie systems. The DLM theory also applies to conservative systems of 
equations, in which case the regular Rankine-Hugoniot jump relation is recovered by letting matrix $A(U)$ be the
jacobian matrix. In other words, the case of conservative systems can be seen as a subset of the non-
conservative systems of equations. 

Another approach, followed in this paper, consists in considering the 
vanishing viscosity solution of \eqt{eq:nc-syst-eq} by adding a parabolic viscous regularization as follows:
%
\begin{equation}
\label{eq:nc-syst-eq-visc}
\partial_t U^\epsilon + A(U^\epsilon) \partial_x U^\epsilon = \epsilon \partial_{xx} U^\epsilon \ , \nonumber
\end{equation}
%
where $\epsilon$ is a vanishing viscosity coefficient. This approach was first studied by Bianchini et al. (REF) and 
further generalized by Alouges et al. (REF). In their work, they show that the solution to the 
vanishing viscosity system is unique as $\epsilon \to 0$ (Theorem 1, page 229 of (REF)). Moreover, Le 
Floch in (REF) generalized the notion of entropy condition to non-conservative system of equations, in the space of 
bounded functions of bounded variation, by introducing an entropy inequality in order to ensure convergence of the 
numerical solution to the entropy weak solution. 
%
This generalization of the entropy inequality to non-conservative hyperbolic systems of equations is of great interest 
for the topic of this paper.
% in order to extend the application of the Entropy Viscosity Method (EVM) to the GRH.
In \cite{our_jcp_radhy_paper}, Delchini et al. proposed to solve the non-equilibrium Grey Radiation-Hydrodynamics 
equations by stabilizing the numerical discretization using the the Entropy Viscosity Method (EVM), 
stabilization is achieved by adding appropriate dissipation terms (viscous fluxes) to the governing laws while ensuring 
that the entropy minimum principle holds. 
%This requires an entropy inequality (we define later in the introduction what we mean by \emph{entropy minimum principle}). 
The viscosity coefficients modulate the magnitude of the added dissipation such that it is large in shock regions 
and vanishingly small elsewhere. In doing so, shocks can be detected and tracked and an adequate amount of viscosity 
is added locally to stabilize the numerical scheme. 
The entropy viscosity coefficients are taken proportional to the entropy production while, at the same time, 
being bounded from above by a first-order viscosity coefficient.
Because of the similarity between Euler equations and the radiation-hydrodynamic equations, it was conjectured in 
\cite{our_jcp_radhy_paper} that the entropy 
viscosity method may be a good candidate for resolving shocks occurring in radiation-hydrodynamic phenomena. Moreover,
the EVM was shown to conserve the ED limit with well-scaled dissipative terms. The ED limit is obtained by performing a 
Chapman-Enskog expansion (REF) of the GRH equations and
taking the leading-order equations. They physically correspond to a material optically thick. The yielding system of equations 
can be written in a \emph{conservative form} with a modified equation of state function of a total pressure defined as the 
sum of hydrodynamic and radiation pressures (REF). When solving the GRH using a numerical method that adds artificial dissipation, 
an asymptotic study needs to be performed in order to make sure that the ED limit is still recovered with well-scaled dissipative terms (REF).

The crux of the EVM lies in satisfying the entropy minimum principle, which states that, at any entropy spatial 
minimum, the rate of entropy change of a system must be positive. When
adding dissipative terms to the governing laws, one must verify that the entropy minimum principle still holds in 
order to single out the entropy solution. Indeed, the functional form of these dissipation terms
should be chosen such that this principle is verified. In \cite{our_jcp_radhy_paper}, we followed an approach 
similar to that of \cite{Balsara, LowrieMorel} 
where the relaxation and diffusion terms in the radiation and material energy equations were omitted in order to 
analyze only the hyperbolic parts of 
the non-equilibrium Grey Radiation-Hydrodynamic equations. Numerical results showed that the functional form of the 
viscous dissipation hence obtained yielded satisfactory results. 

The aim of this paper is two-fold. First, we enhance the theoretical grounds for using the EVM to solve the 
Radiation-Hydrodynamic equations by demonstrating that the omitted diffusion
and relaxation terms do not negatively affect the previously defined dissipative terms and thus still satisfies the 
entropy condition for non-conservative system of equations as stated in REF. Second, we present numerical
results for temperature-dependent opacities and provide convergence rates using semi-analytical solutions.

The remainder of the paper is as follows: in \sct{sec:1d-grh}, we recall
the non-equilibrium Grey Radiation-Hydrodynamic (GRH) equations along with its mathematical properties and some 
theoretical aspects pertaining to non-conservative systems of equations. 
The viscous regularization of the GRH equations based on their hyperbolic parts is recalled in \sct{sec:VR_old}.
In \sct{sec:VR_new}, we demonstrate how that the previous viscous regularization still satisfies the entropy minimum 
principle when one considers the full GRH equations.
After describing discretization and solution techniques in \sct{sec:discr}, 1-D numerical results for Mach 1.05 and 
3 are given in \sct{sec:num-rslt} and compared against semi-analytical solutions obtained from REF. Test cases with 
constant and temperature-dependent opacities are considered. 
Convergence rates of the L$_1$ and L$_2$ norms are also presented for smooth numerical solutions 
(Mach 1.05) and numerical solutions with steady-state shocks (Mach 3).
\tcb{add something in the introduction about the ED limit.}\tcr{yes}
%
%
\vspace{-6pt}
%
%=========================
%=========================
\section{The 1-D Non-Equilibrium Grey Radiation Equations}\label{sec:1d-grh}
%=========================
%=========================
%
\vspace{-2pt}
%
%
In this section, we provide some theoretical background and recall the main results derived in \cite{our_jcp_radhy_paper}. 
These will serve as the foundation for the 
application of the viscous regularization to the full non-equilibrium grey Radiation-Hydrodynamic equations, 
given subsequently  in \sct{sec:VR_new}. First, the 1-D  non-equilibrium GRH equations \emph{without} viscous regularization are given in \eqt{eq:GRH} along with some 
mathematical properties of the convective systems (first-order derivative terms). We also recall theorems of interest for this paper 
related to weak and entropy solutions for non-conservative system of equations. Then, the reader is guided through the main steps 
that lead to the viscous regularization of \cite{our_jcp_radhy_paper} using an 
entropy condition. Finally, the 1-D non-equilibrium grey Radiation-Hydrodynamic equation \emph{with} viscous regularization derived in 
\cite{our_jcp_radhy_paper} are recalled.

%------------------------------------------------------------------
\subsection{The governing equations}\label{sec:GRH-inviscid}
%------------------------------------------------------------------
The 1-D non-equilibrium grey Radiation-Hydrodynamic equations \emph{without} viscous regularization are as follow for any time $t \geq 0$ and spatial coordinate $x \in \mathcal{R}$:
%
\begin{subequations}\label{eq:GRH}
%
\begin{align}
\label{eq:GRHmass}
\partial_t \left( \rho \right) + \partial_x\left( \rho u \right) = 0 \ ,
\end{align}
%
\begin{align}
\label{eq:GRHmom}
\partial_t \left( \rho u\right) + \partial_x \left(\rho u^2 + P + \frac{\epsilon}{3} \right) = 0 \ ,
\end{align}
%
\begin{align}
\label{eq:GRHenerg}
\partial_t \left( \rho E\right) + \partial_x \left[ u \left( \rho E + P \right) \right] = -\frac{u}{3} \partial_x \epsilon - \sigma_a c \left( a T^4 - \epsilon \right) \ ,
\end{align}
%
\begin{align}
\label{eq:GRHrad}
\partial_t \epsilon + \frac{4}{3} \partial_x \left( u \epsilon \right) = \frac{u}{3} \partial_x \epsilon + \partial_x \left( \frac{c}{3 \sigma_t} \partial_x \epsilon \right) 
+ \sigma_a c \left( a T^4 - \epsilon \right) \ ,
\end{align}
%
\begin{align}
\label{eq:EOS}
P = eos(\rho, e) \ ,
\end{align}
\end{subequations}
%
where $\rho$, $\rho u$ and $\rho E$ are the material density, material momentum, and material total energy, 
respectively, and $\epsilon$ denotes the radiation energy density. $P$ and $T$ denote the material pressure and 
material temperature, respectively.  In order to close the system of equations given in \eqt{eq:GRH} 
(the system consists of six variables and four partial differential equations), two closure relationships are needed:  
an equation of state, shown as $eos$ in \eqt{eq:EOS}, relates the material pressure $P$ to the material density $\rho$ and the 
material specific internal energy $e = E - \tfrac 1 2 u^2$. A second equation of state, linking material temperature 
and internal energy, is required to compute the relaxation source term $\sigma_a c \left( a T^4 - \epsilon \right)$.
The variables $c$ and $\sigma_a$ denote the speed of light and the absorption opacity, respectively, and $a$ is the radiation 
constant. The radiation energy density equation, \eqt{eq:GRHrad}, contains a diffusion term that is function of the 
total opacity $\sigma_t$. Note that the opacities $\sigma_a$ and $\sigma_t$ can be either constant or function of 
material properties, typically material density and temperature. 
The theoretical results obtained in this paper are independent of the functional form of the equations of state 
(as long as a concave entropy with respect to $e$ and $1/ \rho$ is considered). The Ideal Gas Equations of State
(IGEOS), however, will be employed in the remainder for simplicity, with the relationships given in \eqt{eq:IGEOS} for the material pressure and temperature variables:
%
\begin{align}\label{eq:IGEOS}
P = (\gamma-1) \rho e \text{  and  } e = C_v T \, ,
\end{align}
%
where the heat ratio $\gamma$ and the heat capacity $C_v$ are assumed constant.

When omitting the relaxation and the diffusion terms from the GRH equations, a first-order partial differential system 
of equations is obtained; this system is hyperbolic since (i) it admits four eigenvalues that are unconditionally real as long as the square of the speed of sound, $c_s$, remains positive ($\lambda_1 = u - c_s$, $\lambda_{2,3} = u$ and  $\lambda_4 = u + c_s$), and (ii) possesses a set of four linearly independent eigenvectors (REF).  
\tcr{you need to find another symbol for the speed of sound. you already used $c$ for the speed of light ...}
Assuming a smooth solution, the following entropy residual equation can be derived from \eqt{eq:GRH}
\tcr{just to be clear, this equation hold for the FULL but UNREGULARIZED GRH, right? I was a bit confused
with the progression of what you want to do here, so maybe we should emphasize this here}
%
\begin{equation}\label{eq:GRH-entropy}
R_e(x,t) = \frac{D s}{Dt} = \left( \partial_t s + u \partial_x s \right) = 0
\end{equation}
%
where the entropy variable is a function of the material density, the material specific internal energy, 
and the radiation density energy, i.e., $s(\rho, e, \epsilon)$. The steps leading to \eqt{eq:GRH-entropy} 
can be found in \cite{our_jcp_radhy_paper}.
We emphasize that \eqt{eq:GRH-entropy} is only valid for smooth numerical solutions and that a different version 
will be presented in \sct{sec:VR_old} for numerical solutions with discontinuities. 
Hereafter, the left-hand-side of \eqt{sec:VR_old}, $R_e$, will be referred to as the \emph{entropy residual}.
%
%
\vspace{-6pt}
%
%=========================
%=========================
\section{The Three Golden Rules}
%=========================
%=========================
%
\vspace{-2pt}
%
%
Before we proceed, we would like to stress \emph{three golden
rules} that need to be followed to enable the most efficient use
of your code at the typesetting stage:
\begin{enumerate}
\item[(i)] keep your own macros to an absolute minimum;

\item[(ii)] as \TeX\ is designed to make sensible spacing
decisions by itself, do \emph{not} use explicit horizontal or
vertical spacing commands, except in a few accepted (mostly
mathematical) situations, such as \verb"\," before a
differential~d, or \verb"\quad" to separate an equation from its
qualifier;

\item[(iii)] follow the \emph{\journalnamelc} reference style.
\end{enumerate}

\pagebreak

\section{Getting Started} The \textsf{\journalclassshort} class file should run
on any standard \LaTeXe\ installation. If any of the fonts, style
files or packages it requires are missing from your installation,
they can be found on the \emph{\TeX\ Collection} DVDs or from
CTAN.

\emph{\journalnamelc} is published using Times fonts and this is
achieved by using the \verb"times"
option as\\
\verb"\documentclass[times]{fldauth}".

\noindent If for any reason you have a problem using Times you can
easily resort to Computer Modern fonts by removing the
\verb"times" option.

\begin{figure}
\setlength{\fboxsep}{0pt}%
\setlength{\fboxrule}{0pt}%
\begin{center}
\begin{boxedverbatim}
\documentclass[times]{fldauth}
%\documentclass[times,doublespace]{fldauth}%For paper submission

\begin{document}

\runningheads{<Initials and Surnames>}{<Short title>}

\title{<Initial cap, lower case>}

\author{<An Author\affil{1},
Someone Else\affil{2}\corrauth\ and Perhaps Another\affil{1}>}

\address{<\affilnum{1}First author's address
(in this example it is the same as the third author)\break
\affilnum{2}Second author's address>}

\corraddr{<Corresponding author's address (the second author in
this example)>. E-mail: <corresponding author's email address>}

%\cgs{<Contract/grant sponsor name (no number)>}
%\cgsn{<Contract/grant sponsor name>}{<number>}

\begin{abstract}
<Text>
\end{abstract}

\keywords{<List keywords>}

\maketitle

\section{Introduction}
.
.
.
\end{boxedverbatim}
\end{center}
%\vspace{-12pt}
\caption{Example header text.\label{F1}}
\end{figure}


\section{The Article Header Information}
The heading for any file using \textsf{\journalclass} is shown in
Figure~\ref{F1}.

\subsection{Remarks}
\begin{enumerate}
\item[(i)] In \verb"\runningheads" use `\emph{et~al.}' if there
are three or more authors.

\item[(ii)] Note the use of \verb"\affil" and \verb"\affilnum" to
link names and addresses. The author for correspondence is marked
by \verb"\corrauth" and \verb"\corraddr" is used to give that
author's address, which will be printed as a footnote, prefaced by
`Correspondence to:'.

\item[(iii)] For submitting a double-spaced manuscript, add
\verb"doublespace" as an option to the documentclass line.

\item[(iv)] Use \verb"\cgs" for giving details of financial
sponsors; alternatively use \verb"\cgsn" if the grant number is
also to be included. These details will be printed as a footnote,
with `Contract/grant sponsor:' and `contract/grant number:'
inserted in the appropriate places.

\item[(v)] The abstract should be capable of standing by itself,
in the absence of the body of the article and of the bibliography.
Therefore, it must not contain any reference citations.

\item[(vi)] Keywords are separated by semicolons.
\end{enumerate}

\begin{figure}
\setlength{\fboxsep}{0pt}%
\setlength{\fboxrule}{0pt}%
\begin{center}
\begin{boxedverbatim}
\begin{table}
\caption{<Table caption>}
\centering
\tabsize
\begin{tabular}{<table alignment>}
\toprule
<column headings>\\
\midrule
<table entries
(separated by & as usual)>\\
<table entries>\\
.
.
.\\
\bottomrule
\end{tabular}
\end{table}
\end{boxedverbatim}
\end{center}
\vspace{-6pt}
\caption{Example table layout.\label{F2}}
\vspace{-6pt}
\end{figure}

\section{The Body of the Article}

\subsection{Mathematics} \textsf{\journalclass} makes the full
functionality of \AmS\/\TeX\ available. We encourage the use of
the \verb"align", \verb"gather" and \verb"multline" environments
for displayed mathematics. \textsf{amsthm} is used for setting
theorem-like and proof environments. The usual \verb"\newtheorem"
command needs to be used to set up the environments for your
particular document.

\subsection{Figures and Tables} \textsf{\journalclass} includes the
\textsf{graphicx} package for handling figures.

Figures are called in as follows:
\begin{verbatim}
\begin{figure}
\centering
\includegraphics{<figure name>}
\caption{<Figure caption>}
\end{figure}
\end{verbatim}

For further details on how to size figures, etc., with the
\textsf{graphicx} package see, for example, \cite{R1}
or \cite{R3}. If figures are available in an
acceptable format (for example, .eps, .ps) they will be used but a
printed version should always be provided. \medbreak

The standard coding for a table is shown in Figure~\ref{F2}.

\subsection{Cross-referencing}
The use of the \LaTeX\ cross-reference system
for figures, tables, equations, etc., is encouraged
(using \verb"\ref{<name>}" and \verb"\label{<name>}").

\subsection{Acknowledgements} An Acknowledgements section is started with \verb"\ack" or
\verb"\acks" for \textit{Acknowledgement} or
\textit{Acknowledgements}, respectively. It must be placed just
before the References.

\subsection{Bibliography}
The normal commands for producing the reference list are:
\begin{verbatim}
\begin{thebibliography}{99}
\bibitem{<x-ref label>}
         <Reference details>
.
.
.
\end{thebibliography}
\end{verbatim}
where \verb"\bibitem{x-ref label}"
corresponds to \verb"\cite{x-ref label}" in the body of the article
and \verb"{99}" is the widest such number expected and determines
the width of the number column in the reference list.

Please note that the file \textsf{wileyj.bst} is available from
the same download page for those authors using \BibTeX.

\subsection{Double Spacing}
If you need to double space your document for submission please
use the \verb+doublespace+ option as shown in the sample layout in
Figure~\ref{F1}.

\section{Support for \textsf{\journalclass}}
We offer on-line support to participating authors. Please contact
us via e-mail at\\
\href{mailto:fldauth-cls@wiley.co.uk}{\texttt{fldauth-cls@wiley.co.uk}}.

We would welcome any feedback, positive or otherwise, on your
experiences of using \textsf{\journalclass}.

\section{Copyright Statement}
Please  be  aware that the use of  this \LaTeXe\ class file is
governed by the following conditions.

\subsection{Copyright}
Copyright \copyright\ \volumeyear\ John Wiley \& Sons, Ltd, The
Atrium, Southern Gate, Chichester, West Sussex, PO19~8SQ, UK. All
rights reserved.

\subsection{Rules of Use}
This class file is made available for use by authors who wish to
prepare an article for publication in the \emph{\journalnamelc}
published by John Wiley \& Sons, Ltd. The user may not exploit any
part of the class file commercially.

This class file is provided on an \emph{as is}  basis, without
warranties of any kind, either express or implied, including but
not limited to warranties of title, or implied  warranties of
merchantablility or fitness for a particular purpose. There will
be no duty on the author[s] of the software or  John Wiley \&
Sons, Ltd to correct any errors or defects in the software. Any
statutory  rights you may have remain unaffected by your
acceptance of these rules of use.

\ack This class file was developed by Sunrise Setting Ltd,
Torquay, Devon, UK. Website:\\
\href{http://www.sunrise-setting.co.uk}{\texttt{www.sunrise-setting.co.uk}}

\begin{thebibliography}{9}

\bibitem{R1} Kopka~H, Daly~PW. 2003. \emph{A Guide to \LaTeX} (4th~edn).
Addison-Wesley.

\bibitem{R2} Lamport~L. 1994. \emph{\LaTeX: a Document Preparation System} (2nd~edn).
Addison-Wesley.

\bibitem{R3} Mittelbach~F, Goossens~M. 2004. \emph{The \LaTeX\ Companion}
(2nd~edn). Addison-Wesley.

\bibliography{mybibfile}
\end{thebibliography}
\end{document}
