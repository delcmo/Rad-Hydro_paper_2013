%\documentclass[times]{fldauth}
\documentclass[times,doublespace]{fldauth}%For paper submission

% \usepackage[dvips,colorlinks,bookmarksopen,bookmarksnumbered,citecolor=red,urlcolor=red]{hyperref}
\usepackage[colorlinks,bookmarksopen,bookmarksnumbered,citecolor=red,urlcolor=red]{hyperref}

\newcommand\BibTeX{{\rmfamily B\kern-.05em \textsc{i\kern-.025em b}\kern-.08em
T\kern-.1667em\lower.7ex\hbox{E}\kern-.125emX}}

\newcommand{\sdd}{\ensuremath{\rho^{-2}\partial_{\rho}(\rho^2 \partial_{\rho} s)}}
\newcommand{\sddhat}{\ensuremath{\hat{\rho}^{-2}\partial_{\hat{\rho}}(\hat{\rho}^2 \partial_{\hat{\rho}} \hat{s})}}
\newcommand{\sde}{\ensuremath{\partial_{\rho e} s}}
\newcommand{\see}{\ensuremath{\partial_{e e} s}}
\newcommand{\srr}{\ensuremath{\partial_{\epsilon \epsilon} s}}

%%%%%%%%%%%%%%%%%%%%%%%
%%%%%%%%%%%%%%%%%%%%%%%
\usepackage{amsmath}
\usepackage{amsthm}
\usepackage{amsfonts}
\usepackage{amsfonts}
\usepackage{color}
\usepackage{setspace}
%\usepackage{listings}
%\lstset{language=C++}
%\usepackage{lscape} 
\usepackage{float}
\usepackage{graphicx}
\usepackage{caption}
\usepackage{subcaption}
\usepackage[titletoc,toc]{appendix}
\usepackage{xspace}
%\usepackage{color}
%\textwidth = 450pt
\def\fxnote#1{\marginpar{\textcolor{green}{#1}}}
\def\fxwarning#1{\marginpar{\textcolor{red}{#1}}}
%\usepackage[a4paper]{geometry}
%
%=================================================================================================
% new commands
% +++++++++++++++++++++++++++++++++++++++++++++++++++++++++++++++++++++++++++++++++++++++++++++++++
\newcommand{\nc}{\newcommand}

% operators
\renewcommand{\div}{\mbold{\nabla}\! \cdot \!}
\newcommand{\grad}{\mbold{\nabla}}
\newcommand{\divv}[1]{\boldsymbol{\nabla}^{#1}\! \cdot \!}
\newcommand{\gradd}[1]{\mbold{\nabla}^{#1}}
\newcommand{\mbold}[1]{\boldsymbol#1}
% latex shortcuts
\newcommand{\bea}{\begin{eqnarray}}
\newcommand{\eea}{\end{eqnarray}}
\newcommand{\be}{\begin{equation}}
\newcommand{\ee}{\end{equation}}
\newcommand{\bal}{\begin{align}}
\newcommand{\eali}{\end{align}}
\newcommand{\bi}{\begin{itemize}}
\newcommand{\ei}{\end{itemize}}
\newcommand{\ben}{\begin{enumerate}}
\newcommand{\een}{\end{enumerate}}
\usepackage{amsthm}
\newtheorem*{remark}{Remark}
% DGFEM commands
\newcommand{\jmp}[1]{[\![#1]\!]}                     % jump
\newcommand{\mvl}[1]{\{\!\!\{#1\}\!\!\}}             % mean value
\newcommand{\keff}{\ensuremath{k_{\textit{eff}}}\xspace}
% shortcut for domain notation
\newcommand{\D}{\mathcal{D}}
% vector shortcuts
\newcommand{\vo}{\mbold{\Omega}}
\newcommand{\vr}{\mbold{r}}
\newcommand{\vn}{\mbold{n}}
\newcommand{\vnk}{\mbold{\mathbf{n}}}
\newcommand{\vj}{\mbold{J}}
\newcommand{\eig}[1]{\| #1 \|_2}
%
\newcommand{\EI}{\mathcal{E}_h^i}
\newcommand{\ED}{\mathcal{E}_h^{\partial \D^d}}
\newcommand{\EN}{\mathcal{E}_h^{\partial \D^n}}
\newcommand{\ER}{\mathcal{E}_h^{\partial \D^r}}
\newcommand{\reg}{\textit{reg}}
%
\newcommand{\norm}{\textrm{norm}}
\renewcommand{\Re}{\textrm{Re}}
\newcommand{\Pe}{\textrm{P\'e}}
\renewcommand{\Pr}{\textrm{Pr}}
%
\newcommand{\resi}{R}
%\newcommand{\resinew}{\tilde{D}_e}
\newcommand{\resinew}{\widetilde{\resi}}
\newcommand{\resisource}{\widetilde{\resi}^{source}}
\newcommand{\matder}[1]{\frac{\textrm{D} #1}{\textrm{D} t}}
%
% extra space
\newcommand{\qq}{\quad\quad}
% common reference commands
\newcommand{\eqt}[1]{Eq.~(\ref{#1})}                     % equation
\newcommand{\fig}[1]{Fig.~\ref{#1}}                      % figure
\newcommand{\tbl}[1]{Table~\ref{#1}}                     % table
\newcommand{\sct}[1]{Section~\ref{#1}}                   % section
\newcommand{\app}[1]{Appendix~\ref{#1}}                   % appendix
%
\newcommand{\ie}{i.e.,\@\xspace}
\newcommand{\eg}{e.g.,\@\xspace}
\newcommand{\psc}[1]{{\sc {#1}}}
\newcommand{\rs}{\psc{R7}\xspace}
%
\newcommand\br{\mathbf{r}}
%\newcommand{\tf}{\varphi}
\newcommand{\tf}{b}
%
\newcommand{\tcr}[1]{\textcolor{red}{#1}}
\newcommand{\tcb}[1]{\textcolor{blue}{#1}}
\newcommand{\mt}[1]{\marginpar{ {\tiny #1}}}
\bibliographystyle{wileyj}

\begin{document}

\runningheads{M.~O.~Delchini, J.~C.~Ragusa}{Change of variable in the quadratic form}

\title{Viscous Regularization of the non-equilibrium Grey Radiation-Hydrodynamics}

\author{Marc O. Delchini\affil{1}, Jean C. Ragusa\corrauth\affil{1}}

\address{\affilnum{1}Department of Nuclear Engineering, Texas A\&M University, College Station, TX 77843, USA}

\corraddr{Department of Nuclear Engineering, Texas A\&M University, College Station, TX 77843, USA. E-mail: jean.ragusa@tamu.edu}

\maketitle

This document aims at detailing the steps for a change of variable in the quadratic from obtained when deriving the entropy equation from the regularized non-equilibrium Radiation-Hydrodynamic equations. The quadratic from is expressed under the form $X^T A X$ where $X=\left[ \partial_x \rho, \partial_x e,  \partial_x \epsilon \right]$ is a vector function of the gradient of the material density, the material internal energy and the radiation energy density, and $A$ is a symmetric matrix as follows:
%
 \begin{equation}
 A = 
\begin{bmatrix}
\rho^{-2}\partial_{\rho} \left( \rho^2 \partial_{\rho} s \right) & \partial_{\rho,e} s & \partial_{\rho} \left( \rho \partial_{\epsilon} s \right) \\
 \partial_{\rho,e} s & \partial_{e,e} s & \partial_{e,\epsilon} s \\
 \partial_{\rho} \left( \rho \partial_{\epsilon} s \right) & \partial_{e,\epsilon} s & \partial_{\epsilon,\epsilon} s
\end{bmatrix}
\,.
\end{equation}
%
In the equilibrium diffusion limit, the variable of interest when deriving the entropy residual are no longer $\left[\rho, e, \epsilon \right]$ but $\left[ \rho, e^*, \epsilon \right]$ where $e^* = e + \frac{aT^4}{\rho}$ and $T$ are the radiation-modified internal energy and material temperature, respectively. We then propose the following change of variable:
%
\begin{equation}\label{eq:chge_var}
X=\left[ \partial_x \rho, \partial_x e,  \partial_x \epsilon \right] \rightarrow Y=\left[ \partial_x \rho, \partial_x \hat{e},  \partial_x \epsilon \right], 
\end{equation}
%
where $\hat{e} = e + \frac{\epsilon}{\rho}$. Note that in the equilibrium diffusion limit, $\epsilon = a T^4$ in the leading- and first-order which yields $\hat{e} = e^* = e + \frac{a T^4}{\rho}$.

Let assume that the Jacobian matrix associated to the change of variable proposed in \eqt{eq:chge_var} is denoted by $P$, i.e. $X=PY$, then the quadratic from in the new basis is derived as follows:
%
\begin{equation}
X^T A X =  (PY)^T A (PY) = Y^T (P^T A P) Y = Y^T \hat{A} Y \, ,
\end{equation}
%
where $\hat{A} = P^T A P$ is now the matrix of interest.

The first task consists in deriving the Jacobian matrix $P$:
%
\begin{equation}
P = \frac{\partial (\partial_x \rho, \partial_x e, \partial_x \epsilon)}{\partial (\partial_x \rho, \partial_x \hat{e}, \partial_x \epsilon)} = 
\begin{bmatrix}
\partial_{\partial_x \rho} \partial_x \rho & \partial_{\partial_x \hat{e}} \partial_x \rho & \partial_{\partial_x \epsilon} \partial_x \rho \\
\partial_{\partial_x \rho} e & \partial_{\partial_x \hat{e}} e & \partial_{\partial_x \epsilon} e \\
\partial_{\partial_x \rho} \partial_x \epsilon & \partial_{\partial_x \hat{e}} \partial_x \epsilon & \partial_{\partial_x \epsilon} \partial_x \epsilon
\end{bmatrix}
\end{equation}
%
From the definition of $\hat{e}$, it is derived that $\partial_x e = \partial_x \hat{e} - \frac{1}{\rho} \partial_x \epsilon + \frac{\epsilon}{\rho^2} \partial_x \rho$ which yields the following expression for the matrix $P$ and its transpose:
%
\begin{equation}\label{eq:jac}
P =
\begin{bmatrix}
1 & 0 & 0 \\
\frac{\epsilon}{\rho^2} & 1 & \frac{-1}{\rho} \\
0 & 0 & 1
\end{bmatrix}
\text{ and }
P^T =
\begin{bmatrix}
1 & \frac{\epsilon}{\rho^2} & 0 \\
0 & 1 & 0 \\
0 & \frac{-1}{\rho} & 1
\end{bmatrix}
\end{equation}
%
To simplify the derivation, the entropy functional form derived from the IGEOS is used:
%
\begin{equation}\label{eq:entropy}
s(\rho, e, \epsilon) = C_v \ln \left( \frac{e}{\rho^{\gamma-1}} \right) + \frac{4}{3} \frac{a^\frac{1}{4}}{\rho} \epsilon^\frac{3}{4} \, .
\end{equation}
%
Then, using the functional form of the entropy provided in \eqt{eq:entropy}, the matrix $A$ becomes:
%
\begin{equation}
A = 
\begin{bmatrix}
-\frac{C_v (\gamma-1)}{\rho^2} & 0 & 0 \\
0 & -\frac{C_v}{e^2} & 0 \\
0 & 0 & -\frac{a^\frac{1}{4}}{4\rho}\epsilon^\frac{-5}{4}
\end{bmatrix}
\end{equation}
%
We now compute $\hat{A}$ by first computing $Z=AP$ and then $\hat{A} = P^T Z = P^T A P$.
%
\begin{equation}
Z = 
\begin{bmatrix}
-\frac{C_v (\gamma-1)}{\rho^2} & 0 & 0 \\
-\frac{\epsilon C_v}{(\rho e)^2} & -\frac{C_v}{e^2} & \frac{C_v}{\rho e^2} \\
0 & 0 & -\frac{a^\frac{1}{4}}{4\rho}\epsilon^\frac{-5}{4}
\end{bmatrix} \, ,
\end{equation}
%
and 
%
\begin{equation}
\hat{A} = 
\begin{bmatrix}
-\frac{C_v}{\rho^2} \left((\gamma-1)+\left(\frac{\epsilon}{\rho e}\right)^2\right) & -\frac{\epsilon C_v}{(\rho e^2)} & \frac{C_v \epsilon}{\rho^3 e^2} \\
-\frac{\epsilon C_v}{(\rho e)^2} & -\frac{C_v}{e^2} & \frac{C_v}{\rho e^2} \\
\frac{\epsilon C_v}{\rho^3 e^2} & \frac{C_v}{\rho e^2} & -\frac{C_v}{(\rho e)^2}-\frac{a^\frac{1}{4}}{4\rho}\epsilon^\frac{-5}{4}
\end{bmatrix} \, .
\end{equation}

\textcolor{red}{
\begin{equation}\label{eq:gen-matrix}
\hat{A} = 
\begin{bmatrix}
\sdd + \sde\frac{2 \epsilon}{\rho^2} + \see \left(\frac{\epsilon}{\rho^2}\right)^2 &
\sde  + \see \left(\frac{\epsilon}{\rho^2}\right) &
-\frac{1}{\rho} \left( \sde  + \see \left(\frac{\epsilon}{\rho^2}\right) \right) \\
\sde  + \see \left(\frac{\epsilon}{\rho^2}\right) &
\see &
-\frac{\see}{\rho} \\
-\frac{1}{\rho} \left( \sde  + \see \left(\frac{\epsilon}{\rho^2}\right) \right) &
-\frac{\see}{\rho} &
\frac{\see}{\rho^2} + \srr
\end{bmatrix} \, .
\end{equation}
}
\tcb{I double checked and have the same expression for the matrix $\hat{A}$ given in \eqt{eq:gen-matrix} after correcting the first entry: it is $\rho^{-2}$ instead of $\rho^2$.}

The matrix $\hat{A}$ is symmetric and needs to be compared against the matrix $A_{EDL}$ of the quadratic form obtained when considering the Equilibrium-Diffusion limit equations (leading-order continuity, momentum and total (material and radiation) energy equation and the second-order radiation energy density equation):
%
 \begin{equation}
 A _{EDL}= 
\begin{bmatrix}
\rho^{-2}\partial_{\rho} \left( \rho^2 \partial_{\rho} s \right) & \partial_{\rho,e^*} s & \partial_{\rho} \left( \rho \partial_{\epsilon} s \right) \\
 \partial_{\rho,e^*} s & \partial_{e^*,e^*} s & \partial_{e^*,\epsilon} s \\
 \partial_{\rho} \left( \rho \partial_{\epsilon} s \right) & \partial_{e^*,\epsilon} s & \partial_{\epsilon,\epsilon} s
\end{bmatrix}
\,.
\end{equation}
%
Note that to obtain the expression for $A_{EDL}$, the second-order radiation energy density is accounted for even if $\epsilon = aT^4$ at leading- and first-order.

Using the functional form of the entropy expressed as a function of $(\rho, e^*, \epsilon)$, 
%
\begin{equation}
s(\rho, \hat{e}, \epsilon) = C_v \ln \left( \frac{\hat{e}-\frac{\epsilon}{\rho}}{\rho^(\gamma-1)} \right) + \frac{4}{3} \frac{a^\frac{1}{4}}{\rho} \epsilon^\frac{3}{4} \, ,
\end{equation}
%
it can be shown that $A_{EDL} = \hat{A}$ when assuming $\hat{e} = e^*$. In the EDL, the radiation energy density equation is substituted by the simple relation $\epsilon = a T^4$ which is equivalent to removing the last row and last columns of the matrices $A_{EDL}$ and $\hat{A}$.\\
%
We now want to prove that the matrix $\hat{A}$ devolves into the matrix $A_{EDL}$ when using the functional form of the entropy:
%
\begin{equation}
s(\rho,e ,\epsilon) = s_E ( \rho ,e) + \frac{s_R(\epsilon)}{\rho} \text{ or }  \hat{s}(\rho,\hat{e} ,\epsilon) = \hat{s}_E ( \rho ,\hat{e}) + \frac{s_R(\epsilon)}{\rho}
\end{equation}
%
We need to show that the matrices $\hat{A}$ and $A_{EDL}$ have the same entries in order to prove that they are equal. Before diving into the derivation let recall some basic chain rules results. It is assumed that the following change of variable is performed:
%
\begin{equation}
\hat{f}(x,y,z) \rightarrow f(\alpha, \beta, \gamma) \nonumber \, ,
\end{equation}
%
which leads to the following chain rule when computing the derivative of a function $\hat{f}(x,y,z)$ with respect to $x_i = \{x,y,z \}$:
%
\begin{equation}\label{eq:chain_rule_chge_var}
\partial_{x_i} \hat{f} = \frac{\partial \alpha}{\partial x_i} \partial_\alpha f + \frac{\partial \beta}{\partial x_i} \partial_\beta f + \frac{\partial \gamma}{\partial x_i} \partial_\gamma f \, .
\end{equation}
%
Note that in the case under consideration, $x=\alpha$, $z=\gamma$ and $y=f(\alpha, \beta, \gamma)$ which will lead to some simplifications and confusion in the notation. To improve the readiness of the below proof, we choose to note $(\hat{\rho}, \hat{e}, \hat{\epsilon})$ the set of variables referring to the function $\hat{s}$ and $(\rho,e,\epsilon)$ the set of variables referring to the function $s$. Under these assumptions and remembering that $\hat{e} = e + \frac{\hat{\epsilon}}{\hat{\rho}}$, the following results can be derived:
%
\begin{eqnarray}
\frac{\partial \rho}{\partial \hat{\rho}} = 1 \text{, } \frac{\partial \rho}{\partial \hat{e}} = \frac{\partial \rho}{\partial \hat{\epsilon}} = 0 \nonumber \\
\frac{\partial e}{\partial \hat{\rho}} = \frac{\epsilon}{\rho^2} \text{, } \frac{\partial e}{\partial \hat{e}} = 1 \text{, } \frac{\partial e}{\partial \hat{\epsilon}} = \frac{-1}{\rho} \nonumber \\
\frac{\partial \epsilon}{\partial \hat{\rho}} = \frac{\partial \epsilon}{\partial \hat{e}} = 0 \text{, } \frac{\partial \epsilon}{\partial \hat{\epsilon}} = 1 \nonumber
\end{eqnarray}
%
Let start with the first entry in the top left corner of the matrix $A_{EDL}$:
%
\begin{eqnarray}\label{eq:first_entry_A_hat}
A_{EDL,1,1} = \sddhat& =& \rho^{-2} \partial_{\hat{\rho}} \left[ \rho^2 \left( \frac{\partial \rho}{\partial \hat{\rho}} \partial_\rho s + \frac{\partial e}{\partial \hat{\rho}} \partial_e s + \frac{\partial \epsilon}{\partial \hat{\rho}} \partial_\epsilon s \right) \right] \nonumber \\
\sddhat& =& \rho^{-2} \partial_{\hat{\rho}} \left[ \rho^2 \left( \partial_\rho s + \frac{\epsilon}{\rho^2} \partial_e s \right) \right] \nonumber \\
\sddhat& =& \rho^{-2} \partial_{\hat{\rho}} \left[ \rho^2 \partial_\rho s + \epsilon \partial_e s \right] \nonumber \\
\sddhat& =& \rho^{-2} \left[ 2\rho \partial_\rho s + \rho^2 \left( \partial_{\rho,\rho} s + \frac{\epsilon}{\rho^2} \partial_{\rho,e} s \right) + \epsilon \partial_{\rho,e} s + \frac{\epsilon^2}{\rho^2}\partial_{e,e} s \right] \nonumber \\
\sddhat& =& \rho^{-2} \left[ \sdd + 2 \epsilon \partial_{\rho,e} + \left( \frac{\epsilon}{\rho} \right)^2 \partial_{e,e} s \right] = A_{1,1} \, ,
\end{eqnarray}
%
since $\hat{\rho} = \rho$. The righthand-side of \eqt{eq:first_entry_A_hat} is identical to the first entry in the top right corner of the matrix $\hat{A}$ which yields $\hat{A}_{1,1} = A_{EDL,1,1}$. 

The same procedure is followed for the entry $A_{EDL,1,2}$ of the matrix $A_{EDL}$ as follows:
%
\begin{eqnarray}\label{eq:scd_entry_A_hat}
A_{EDL,1,2} = \partial_{\hat{\rho},\hat{e}} \hat{s}& =& \partial_{\hat{\rho}} \left[ \frac{\partial \rho}{\partial \hat{e}} \partial_\rho s + \frac{\partial e}{\partial \hat{e}} \partial_e s + \frac{\partial \epsilon}{\partial \hat{e}} \partial_\epsilon s \right] \nonumber \\ 
\partial_{\hat{\rho},\hat{e}} \hat{s}& =& \partial_{\hat{\rho}} \left( \partial_e s \right) \nonumber \\
\partial_{\hat{\rho},\hat{e}} \hat{s}& = &\frac{\partial \rho}{\partial \hat{\rho}} \partial_{\rho,e} s + \frac{\partial e}{\partial \hat{\rho}} \partial_{e,e} s + \frac{\partial \epsilon}{\partial \hat{\rho}} \partial_{\epsilon,e} s \nonumber \\
\partial_{\hat{\rho},\hat{e}} \hat{s}& = &\partial_{\rho,e} s + \frac{\epsilon}{\rho^2} \partial_{e,e} s = A_{1,2} \, .
\end{eqnarray}
%
Once again, the entries of the matrices $A_{EDL}$ and A match. The same lengthy procedure can be applied to the other entries of the matrix $A_{EDL}$ to prove that they are equal to the entries of the matrix A. It is then concluded that the matrix $A$ and $A_{EDL}$ are identical under the change of variables $(\rho, e, \epsilon) \rightarrow (\rho, e+ \frac{\epsilon}{\rho}, \epsilon)$.\\
\tcb{I hope this is clear enough.}
\end{document}