\section{Implementation of boundary conditions} \label{App:AppendixA}
This appendix deals with the implementation of the boundary conditions for the Grey Radiation-Hydrodynamic equations. Because of the first- and second-order differential terms, the GRH equations are wave-dominated equations that require the implementation of boundary conditions consistent with the characteristics. 

Study of the hyperbolic term of the GRH equations yield four eigenvalues that are $\lambda_{1,2}=u$ and $\lambda_{3,4}=u \pm c_m$. The sign of the eigenvalues inform us on how the physical information travel in the computational domain and at the boundaries. We only consider in the appendix the cases of a left supersonic and a right subsonic boundaries as they are of interest to the tests presented in this paper. Under the assumption of supersonic left boundary condition, all eigenvalues are positive and none physical information leave the computational domain. As a result, four boundary values must be user-specified and used to compute the boundary fluxes from the advection terms. In the other hand, the treatment of the right subsonic boundary is quite different as three eigenvalues are positive and one is negative, meaning, physical information leaves and enters the computational domain at the right boundary. Consequently, the right boundary fluxes from the advection terms are computed using one user-specified boundary value, typically the material pressure, and three boundary values supplied by the code, i.e. current values of the material density, velocity and radiation energy density. 

The radiation-diffusion equation contains a elliptic term that also yields a boundary term.

In the continuous Galerkin finite element approach, boundary terms arise from the integration per part of first- and second-order conservative terms. We denote by the fluxes F(U) and D(U) the first- and second-order conservative terms in the 1-D GRH equations, respectively:
%
\begin{equation}
F(U) + D(U) = 
\begin{bmatrix}
\rho u \\
\rho u^2 + P + \frac{\epsilon}{3} \\
u \left( \rho E + P \right) \\
\frac{4}{3} u \epsilon
\end{bmatrix}
+  
\begin{bmatrix}
0 \\
0 \\
0 \\
- \frac{c}{3 \sigma_t} \partial_x \epsilon
\end{bmatrix}
\,.
\end{equation}
%
To discretize the conservative flux F(U), one multiplies by a test function $\phi$, integrate over the computational domain and integrate per part. The resulting boundary terms are as follows:
%
\begin{eqnarray}
\left[\left(F(U)+D(U)\right) \cdot \vec{n} \ \phi \right]_{left} - \left[ \left(F(U)+D(U)\right) \cdot \vec{n} \ \phi \right]_{right} \, ,
\end{eqnarray}
%
where $\vec{n}$ is the outward normal. The fluxes at the left and right boundaries are computed from user specified values and from 
%\begin{subequations}
%%
%\begin{equation}
%\label{eq:GRHmass}
%\partial_t \left( \rho \right) + \partial_x\left( \rho u \right) = \partial_x \left( \kappa \partial_x \rho \right) \, ,
%\end{equation}
%%
%\begin{equation}
%\label{eq:GRHmom}
%\partial_t \left( \rho u\right) + \partial_x \left(\rho u^2 + P + \frac{\epsilon}{3} \right) = \partial_x \left( \kappa \partial_x \rho u \right) \, ,
%\end{equation}
%%
%\begin{equation}
%\label{eq:GRHenerg}
%\partial_t \left( \rho E\right) + \partial_x \left[ u \left( \rho E + P \right) \right] + \frac{u}{3} \partial_x \epsilon + \sigma_a c \left( \ar T^4 - \epsilon \right) = \partial_x \left( \kappa \partial_x \rho E \right)\, ,
%\end{equation}
%%
%\begin{equation}
%\label{eq:GRHrad}
%\partial_t \epsilon + \frac{4}{3} \partial_x \left( u \epsilon \right) - \frac{u}{3} \partial_x \epsilon - \partial_x \left( \frac{c}{3 \sigma_t} \partial_x \epsilon \right) 
%- \sigma_a c \left( \ar T^4 - \epsilon \right)  = \partial_x \left( \kappa \partial_x \epsilon \right)\, ,
%\end{equation}
%%
%\end{subequations}