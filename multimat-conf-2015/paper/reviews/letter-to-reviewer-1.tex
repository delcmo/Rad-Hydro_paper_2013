\documentclass{article}
%%%%%%%%%%%%%%%%%%%%%%%%%%%%%%%%%%%%%%%%%%%%%%%%%%%%%%%%%%%%%%%%%%%%%%%%%%%%%%%%%%%%%%%%%%%%%%%%%%%%%%%%%%%%%%%%%%%%%%%%%%%%%%%%%%%%%%%%%%%%%%%%%%%%%%%%%%%%%%%%%%%%%%%%%%%%%%%%%%%%%%%%%%%%%%%%%%%%%%%%%%%%%%%%%%%%%%%%%%%%%%%%%%%%%%%%%%%%%%%%%%%%%%%%%%%%
\usepackage{amsmath,amssymb}
% more math
\usepackage{amsfonts}
\usepackage{amssymb}
\usepackage{amstext}
\usepackage{amsbsy}

\usepackage{color}
\newcommand{\mt}[1]{\marginpar{\small #1}}
%%%%%%%%%%%%%%%%%%%%%%%%%%%%%%%%%%%%%%%%%%%%%%%%%%%%%%%%%%%%%%%%%%%%
% new commands
\newcommand{\nc}{\newcommand}
% operators
\renewcommand{\div}{\vec{\nabla}\! \cdot \!}
\newcommand{\grad}{\vec{\nabla}}
% latex shortcuts
\newcommand{\bea}{\begin{eqnarray}}
\newcommand{\eea}{\end{eqnarray}}
\newcommand{\be}{\begin{equation}}
\newcommand{\ee}{\end{equation}}
\newcommand{\bal}{\begin{align}}
\newcommand{\eali}{\end{align}}
\newcommand{\bi}{\begin{itemize}}
\newcommand{\ei}{\end{itemize}}
\newcommand{\ben}{\begin{enumerate}}
\newcommand{\een}{\end{enumerate}}
% DGFEM commands
\newcommand{\jmp}[1]{[\![#1]\!]}                     % jump
\newcommand{\mvl}[1]{\{\!\!\{#1\}\!\!\}}             % mean value
\newcommand{\keff}{\ensuremath{k_{\textit{eff}}}\xspace}
% shortcut for domain notation
\newcommand{\D}{\mathcal{D}}
% vector shortcuts
\newcommand{\vo}{\vec{\Omega}}
\newcommand{\vr}{\vec{r}}
\newcommand{\vn}{\vec{n}}
\newcommand{\vnk}{\vec{\mathbf{n}}}
\newcommand{\vj}{\vec{J}}
% extra space
\newcommand{\qq}{\quad\quad}
% common reference commands
\newcommand{\eqt}[1]{Eq.~(\ref{#1})}                     % equation
\newcommand{\fig}[1]{Fig.~\ref{#1}}                      % figure
\newcommand{\tbl}[1]{Table~\ref{#1}}                     % table

\newcommand{\ud}{\,\mathrm{d}}

\newcommand{\tcr}[1]{\textcolor{red}{#1}}
\newcommand{\tcb}[1]{\textcolor{blue}{#1}}
\newcommand{\tcg}[1]{\textcolor{green}{#1}}
%%%%%%%%%%%%%%%%%%%%%%%%%%%%%%%%%%%%%%%%%%%%%%%%%%%%%%%%%%%%%%%%%%%%
\bibliographystyle{../wileyj}

\begin{document}

\begin{center}
{ \Large Answers to Reviewer \# 1}
\end{center}

\bigskip

\noindent Manuscript \# FLD-16-0027 \\
Title: ``Regularization of the non-equilibrium Grey Radiation-Hydrodynamic equations with an artificial viscosity method', \\
{\it International Journal for Numerical Methods in Fluids}\\

\bigskip
\bigskip

{\color{blue}
Comments to the Author: \\ 
The manuscript describes an artificial viscosity method to solve a non-conservative hyperbolic model of coupled hydrodynamics and radiation. The manuscript is well written and of good technical level.
Important details are however missing in the present manuscript version.\\}
Thank you.
\bigskip

{\color{blue}
1. The flow model is presented in page 4. Its limitations are not mentioned. What are the assumptions? For which kind of application this model can be used. This is important for the reader. \\}
This set of equations is suitable for nonrelativistic radiation hydrodynamics and was derived in \cite{LowrieMorelHittinger, LowrieMorel}. It is shown that the system of equations is correct to $O(u/c)$ which means that the error term is $O(u^2/c^2)$, where $u$ is the characteristic fluid and $c$ is the speed of light. Also, the set of equations solved in this paper is the simplest nonequilibrium model and assumes a isotropic angular variation of the radiation which yields a constant Eddington factor of 1/3 in the diffusion term. It can be used to verify simulation codes. One of the advantage of this system of equations is that it is hyperbolic and thus,  can be numerically integrated using high-order numerical methods based on a wave decomposition for example. We added these two references in Section 2.1 right before Eq. 2. 
\tcr{for diffusion to be valid, you only need linear anisotropy, not complete isotropic variation. why do you say isotropic??? Also, is this system hyperbolic? It contains a diffusion equation!!!!}
\bigskip

{\color{blue}
2. Page 5, Section 2.3. Because of their hyperbolic nature the GRH equations can develop shocks.
Could you explain/detail what shock structure is expected? One could expect constant radiation through the shock.
If the shock relations are known, they should be presented with System 5 and commented. Page 13 the reader is directed to Refs 24 and 28 meaning these relations are available.\\}
A detailed description of the wave structure for the system of equations of interest can be found in \cite{Balsara}. The jump relations are widely available in the published literature (see \cite{LowrieEdwards, LowrieMorelHittinger} and references therein). The same jump relations are used to compute the initial conditions and we added a paragraph in lines 225-234 detailing the process to follow and the equations to use for this purpose instead of simply referencing the reader to \cite{LowrieEdwards}. The shock problem of interest in this paper consists of an embedded shock with precursor and relaxation regions \cite{LowrieEdwards}.
\tcb{constant radiation through the shock?}\tcr{what are the lines 225-234 in the latest version? }
\bigskip

{\color{blue}
3. Several times the authors refer to Ref 6. Is the method of that reference in agreement with the shock relations of Refs 24-28?
\\}
\cite{dlm} ([6] in the first paper version) provides the mathematical theoretical ground to develop numerical methods for non-conservative system of equations that are also valid for conservative system of equations. The GRH equations without source terms are a non-conservative system of equations and thus the theory developed in \cite{dlm} should be used. Jump relations for the non-conservative system of equations apply to the GRH and to any conservative system of equations, and thus should be in agreement with the ones presented in \cite{LowrieEdwards} and \cite{Toro}.
\bigskip

{\color{blue}
4. Section 6. The results at steady state are compared with an exact solution. It should be presented in an Appendix with illustrations showing the configuration studied as well as boundary conditions. \\}
We added an appendix explaining how the boundary conditions are implemented. We also added in the paper a paragraph explaining how the semi-analytical solutions are obtained and provide references for the reader in lines 235-244.

As for the configuration studied, we believe you are referring to the wave decomposition that is usually used in a Riemann solver. Since we are not using such a solver in the numerical method presented in \cite{our_jcp_radhy_paper} and in this paper, we are just giving references in the introduction.
\bigskip

{\color{blue}
\noindent Minor concerns: \\
\indent 1. Page 4,5 lines from bottom: ?can be derived hyperbolic parts of the GRH? ... please correct.\\}
It is now corrected. Thank you.

\bigskip

{\color{blue}
 2. Rusanov instead of Rusabov.\\}
It is now corrected. Thank you.

\bibliography{../mybibfile}
\end{document}

