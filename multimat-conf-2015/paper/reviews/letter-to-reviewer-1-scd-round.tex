\documentclass{article}
%%%%%%%%%%%%%%%%%%%%%%%%%%%%%%%%%%%%%%%%%%%%%%%%%%%%%%%%%%%%%%%%%%%%%%%%%%%%%%%%%%%%%%%%%%%%%%%%%%%%%%%%%%%%%%%%%%%%%%%%%%%%%%%%%%%%%%%%%%%%%%%%%%%%%%%%%%%%%%%%%%%%%%%%%%%%%%%%%%%%%%%%%%%%%%%%%%%%%%%%%%%%%%%%%%%%%%%%%%%%%%%%%%%%%%%%%%%%%%%%%%%%%%%%%%%%
\usepackage{amsmath,amssymb}
% more math
\usepackage{amsfonts}
\usepackage{amssymb}
\usepackage{amstext}
\usepackage{amsbsy}

\usepackage{color}
\newcommand{\mt}[1]{\marginpar{\small #1}}
%%%%%%%%%%%%%%%%%%%%%%%%%%%%%%%%%%%%%%%%%%%%%%%%%%%%%%%%%%%%%%%%%%%%
% new commands
\newcommand{\nc}{\newcommand}
% operators
\renewcommand{\div}{\vec{\nabla}\! \cdot \!}
\newcommand{\grad}{\vec{\nabla}}
% latex shortcuts
\newcommand{\bea}{\begin{eqnarray}}
\newcommand{\eea}{\end{eqnarray}}
\newcommand{\be}{\begin{equation}}
\newcommand{\ee}{\end{equation}}
\newcommand{\bal}{\begin{align}}
\newcommand{\eali}{\end{align}}
\newcommand{\bi}{\begin{itemize}}
\newcommand{\ei}{\end{itemize}}
\newcommand{\ben}{\begin{enumerate}}
\newcommand{\een}{\end{enumerate}}
% DGFEM commands
\newcommand{\jmp}[1]{[\![#1]\!]}                     % jump
\newcommand{\mvl}[1]{\{\!\!\{#1\}\!\!\}}             % mean value
\newcommand{\keff}{\ensuremath{k_{\textit{eff}}}\xspace}
% shortcut for domain notation
\newcommand{\D}{\mathcal{D}}
% vector shortcuts
\newcommand{\vo}{\vec{\Omega}}
\newcommand{\vr}{\vec{r}}
\newcommand{\vn}{\vec{n}}
\newcommand{\vnk}{\vec{\mathbf{n}}}
\newcommand{\vj}{\vec{J}}
% extra space
\newcommand{\qq}{\quad\quad}
% common reference commands
\newcommand{\eqt}[1]{Eq.~(\ref{#1})}                     % equation
\newcommand{\fig}[1]{Fig.~\ref{#1}}                      % figure
\newcommand{\tbl}[1]{Table~\ref{#1}}                     % table

\newcommand{\ud}{\,\mathrm{d}}

\newcommand{\tcr}[1]{\textcolor{red}{#1}}
\newcommand{\tcb}[1]{\textcolor{blue}{#1}}
\newcommand{\tcg}[1]{\textcolor{green}{#1}}
%%%%%%%%%%%%%%%%%%%%%%%%%%%%%%%%%%%%%%%%%%%%%%%%%%%%%%%%%%%%%%%%%%%%
\bibliographystyle{../wileyj}

\begin{document}

\begin{center}
{ \Large Answers to Reviewer \# 1}
\end{center}

\bigskip

\noindent Manuscript \# FLD-16-0027 \\
Title: ``Regularization of the non-equilibrium Grey Radiation-Hydrodynamic equations with an artificial viscosity method', \\
{\it International Journal for Numerical Methods in Fluids}\\

\bigskip
%{\it The References cited in our letter are provided below, at the end of our replies. They can be found in the revised manuscript as well,
%but with a different citing number (BibTeX renumbers them in each document).}
%\bigskip

{\color{blue}
Comments to the Author: \\ 
1. System (2) being non-conventional the authors may give associated shock conditions. Added details in lines 224-233 do not give much information. Related issues in my former report are not answered by the revised version.\\}
We added a reference in Section 2.1 (lines 111-112) for the shock conditions associated to the GRH. The same jump relations of the un-regularized GRH, i.e.. without artificial viscous fluxes, are recalled for a steady shock below 
(see also Eq.(12) of \cite{LowrieEdwards}):
%
\begin{eqnarray}\label{eq:jump-relations}
& (\rho u )_{left} = (\rho u )_{right} \, , \nonumber \\
& (\rho u^2 + P )_{left} = (\rho u^2 + P )_{right} \, , \nonumber \\
& \left[ u(\rho E + P )\right]_{left} = \left[ u(\rho E + P )\right]_{right} \, , \nonumber \\
& \left[ \frac{4}{3}u \epsilon - \frac{c}{3\sigma_t} \partial_x \epsilon \right]_{left} = \left[ \frac{4}{3}u \epsilon - \frac{c}{3\sigma_t} \partial_x \epsilon \right]_{right} \, , \nonumber
\end{eqnarray}
%
where $left$ and $right$ correspond to the pre- and post-shock states, respectively. Note that $\epsilon$ remains smooth across jump whereas the material properties $\rho$, $u$, $P$ and $T$ may be discontinuous as observed in the numerical results presented in the paper.

\bigskip

{\color{blue}
 2. Lines 20-24 regarding DLM theory of non conservative products. The authors consider that this theory 'must be used to rigorously define, in a weak sense, the non-conservative products ..'. This theory is not general and the author should moderate their considerations. For example it is shown in the following reference that this method is inappropriate:
Abgrall, R., \& Karni, S. (2010). A comment on the computation of non-conservative products. Journal of Computational Physics, 229(8), 2759-2763\\}
We fully agree with you and moderated the introduction by saying that the DLM theory may be applied to non-conservative system of equations but does not guarantee convergence of the numerical solution to the weak solution (see lines 23-25 and lines 53-54). We also included the reference you provided for completeness (line 51).
\bigskip

{\color{blue}
3. Nothing has been added regarding model's validity:
\begin{itemize}
\item is it accurate for flames?
\item is it appropriate for radiation in air or only in dense matter?
\end{itemize}
References (3, 6) do not give that much information.\\}
We added a sentence in the introduction of the manuscript in lines 12-15 regarding the application domains of the GRH with appropriate references.\\

The GRH equations contain ``$u/c$'' terms (e.g., $u/(3c) \partial_x \epsilon$, where I divided by $c$ to make the terminology ``$u/c$'' obvious). 
These terms typically come into play when temperatures exceed the million-degree mark and the material kinetic energy is significantly affected 
by radiation momentum deposition. This is the situation in astrophysical environments and in high-energy high-density experiments, but not necessarily 
in combustion and flame modeling. Moreover, combustion/flame modeling oftentimes needs photon spectral information which requires that frequency-dependent (not grey) 
equations are solved.
% ; here we solve the grey (frequency-integrated) equations. Moreover, 
%The general GRH equations (grey nonequilibrium-diffusion radiation coupled to hydrodynamic equations) are a slightly relaxed version of the equilibrium-diffusion approximation (EDA), which assumes:
%\begin{enumerate}
%\item the photon mean-free-path is small compared to the size of the system (optically thick),
%\item the material and radiation are in thermal equilibrium,
%\item the radiation flux is time-independent diffusive \tcr{what does that mean?} ($\vec{F} \propto - \grad$ E, and if in the lab-frame then add the radiation enthalpy, $4T^4\grad \vec{u} / 3$), \tcr{a coef. is missing between $F$,which, by the way, should be a vector $\vec{F}$. Also, the end of the sentence reads really funny. If in the lab frame, then, by the way, add one more term. It's better to use big picture physics ideas.}
%\item the radiation pressure is isotropic.
%\end{enumerate}
%The obvious distinction between the EDA your GRH equations is that assumption 2) is no longer enforced, but it is still assumed that the system is optically thick.  But this is more a statement about the assumptions that go into these specific equations, than a statement about whether the computational algorithm 'will' (although it might be expected to) produce reasonable solutions for a given physical environment. \tcr{I do not follow how this last sentence is answering one of the reviewer's questions. His comments was about model validity and we seem to embark on a discussion EDA versus GRH} 

\bibliography{../mybibfile}
\end{document}

