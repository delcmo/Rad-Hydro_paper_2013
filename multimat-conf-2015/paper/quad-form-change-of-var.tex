%\documentclass[times]{fldauth}
\documentclass[times,doublespace]{fldauth}%For paper submission

% \usepackage[dvips,colorlinks,bookmarksopen,bookmarksnumbered,citecolor=red,urlcolor=red]{hyperref}
\usepackage[colorlinks,bookmarksopen,bookmarksnumbered,citecolor=red,urlcolor=red]{hyperref}

\newcommand\BibTeX{{\rmfamily B\kern-.05em \textsc{i\kern-.025em b}\kern-.08em
T\kern-.1667em\lower.7ex\hbox{E}\kern-.125emX}}


%%%%%%%%%%%%%%%%%%%%%%%
%%%%%%%%%%%%%%%%%%%%%%%
\usepackage{amsmath}
\usepackage{amsthm}
\usepackage{amsfonts}
\usepackage{amsfonts}
\usepackage{color}
\usepackage{setspace}
%\usepackage{listings}
%\lstset{language=C++}
%\usepackage{lscape} 
\usepackage{float}
\usepackage{graphicx}
\usepackage{caption}
\usepackage{subcaption}
\usepackage[titletoc,toc]{appendix}
\usepackage{xspace}
%\usepackage{color}
%\textwidth = 450pt
\def\fxnote#1{\marginpar{\textcolor{green}{#1}}}
\def\fxwarning#1{\marginpar{\textcolor{red}{#1}}}
%\usepackage[a4paper]{geometry}
%
%=================================================================================================
% new commands
% +++++++++++++++++++++++++++++++++++++++++++++++++++++++++++++++++++++++++++++++++++++++++++++++++
\newcommand{\nc}{\newcommand}

% operators
\renewcommand{\div}{\mbold{\nabla}\! \cdot \!}
\newcommand{\grad}{\mbold{\nabla}}
\newcommand{\divv}[1]{\boldsymbol{\nabla}^{#1}\! \cdot \!}
\newcommand{\gradd}[1]{\mbold{\nabla}^{#1}}
\newcommand{\mbold}[1]{\boldsymbol#1}
% latex shortcuts
\newcommand{\bea}{\begin{eqnarray}}
\newcommand{\eea}{\end{eqnarray}}
\newcommand{\be}{\begin{equation}}
\newcommand{\ee}{\end{equation}}
\newcommand{\bal}{\begin{align}}
\newcommand{\eali}{\end{align}}
\newcommand{\bi}{\begin{itemize}}
\newcommand{\ei}{\end{itemize}}
\newcommand{\ben}{\begin{enumerate}}
\newcommand{\een}{\end{enumerate}}
\usepackage{amsthm}
\newtheorem*{remark}{Remark}
% DGFEM commands
\newcommand{\jmp}[1]{[\![#1]\!]}                     % jump
\newcommand{\mvl}[1]{\{\!\!\{#1\}\!\!\}}             % mean value
\newcommand{\keff}{\ensuremath{k_{\textit{eff}}}\xspace}
% shortcut for domain notation
\newcommand{\D}{\mathcal{D}}
% vector shortcuts
\newcommand{\vo}{\mbold{\Omega}}
\newcommand{\vr}{\mbold{r}}
\newcommand{\vn}{\mbold{n}}
\newcommand{\vnk}{\mbold{\mathbf{n}}}
\newcommand{\vj}{\mbold{J}}
\newcommand{\eig}[1]{\| #1 \|_2}
%
\newcommand{\EI}{\mathcal{E}_h^i}
\newcommand{\ED}{\mathcal{E}_h^{\partial \D^d}}
\newcommand{\EN}{\mathcal{E}_h^{\partial \D^n}}
\newcommand{\ER}{\mathcal{E}_h^{\partial \D^r}}
\newcommand{\reg}{\textit{reg}}
%
\newcommand{\norm}{\textrm{norm}}
\renewcommand{\Re}{\textrm{Re}}
\newcommand{\Pe}{\textrm{P\'e}}
\renewcommand{\Pr}{\textrm{Pr}}
%
\newcommand{\resi}{R}
%\newcommand{\resinew}{\tilde{D}_e}
\newcommand{\resinew}{\widetilde{\resi}}
\newcommand{\resisource}{\widetilde{\resi}^{source}}
\newcommand{\matder}[1]{\frac{\textrm{D} #1}{\textrm{D} t}}
%
% extra space
\newcommand{\qq}{\quad\quad}
% common reference commands
\newcommand{\eqt}[1]{Eq.~(\ref{#1})}                     % equation
\newcommand{\fig}[1]{Fig.~\ref{#1}}                      % figure
\newcommand{\tbl}[1]{Table~\ref{#1}}                     % table
\newcommand{\sct}[1]{Section~\ref{#1}}                   % section
\newcommand{\app}[1]{Appendix~\ref{#1}}                   % appendix
%
\newcommand{\ie}{i.e.,\@\xspace}
\newcommand{\eg}{e.g.,\@\xspace}
\newcommand{\psc}[1]{{\sc {#1}}}
\newcommand{\rs}{\psc{R7}\xspace}
%
\newcommand\br{\mathbf{r}}
%\newcommand{\tf}{\varphi}
\newcommand{\tf}{b}
%
\newcommand{\tcr}[1]{\textcolor{red}{#1}}
\newcommand{\tcb}[1]{\textcolor{blue}{#1}}
\newcommand{\mt}[1]{\marginpar{ {\tiny #1}}}
\bibliographystyle{wileyj}

\begin{document}

\runningheads{M.~O.~Delchini, J.~C.~Ragusa}{Change of variable in the quadratic form}

\title{Viscous Regularization of the non-equilibrium Grey Radiation-Hydrodynamics}

\author{Marc O. Delchini\affil{1}, Jean C. Ragusa\corrauth\affil{1}}

\address{\affilnum{1}Department of Nuclear Engineering, Texas A\&M University, College Station, TX 77843, USA}

\corraddr{Department of Nuclear Engineering, Texas A\&M University, College Station, TX 77843, USA. E-mail: jean.ragusa@tamu.edu}

\maketitle

This document aims at detailing the steps for a change of variable in the quadratic from obtained when deriving the entropy equation from the regularized non-equilibrium Radiation-Hydrodynamic equations. The quadratic from is expressed under the form $X^T A X$ where $X=\left[ \partial_x \rho, \partial_x e,  \partial_x \epsilon \right]$ is a vector function of the gradient of the material density, the material internal energy and the radiation energy density, and $A$ is a symmetric matrix as follows:
%
 \begin{equation}
 A = 
\begin{bmatrix}
\rho^{-2}\partial_{\rho} \left( \rho^2 \partial_{\rho} s \right) & \partial_{\rho,e} s & \partial_{\rho} \left( \rho \partial_{\epsilon} s \right) \\
 \partial_{\rho,e} s & \partial_{e,e} s & \partial_{e,\epsilon} s \\
 \partial_{\rho} \left( \rho \partial_{\epsilon} s \right) & \partial_{e,\epsilon} s & \partial_{\epsilon,\epsilon} s
\end{bmatrix}
\,.
\end{equation}
%
In the equilibrium diffusion limit, the variable of interest when deriving the entropy residual are no longer $\left[\rho, e, \epsilon \right]$ but $\left[ \rho, e^*, \epsilon \right]$ where $e^* = e + \frac{aT^4}{\rho}$ and $T$ are the radiation-modified internal energy and material temperature, respectively. We then propose the following change of variable:
%
\begin{equation}\label{eq:chge_var}
X=\left[ \partial_x \rho, \partial_x e,  \partial_x \epsilon \right] \rightarrow Y=\left[ \partial_x \rho, \partial_x \hat{e},  \partial_x \epsilon \right], 
\end{equation}
%
where $\hat{e} = e + \frac{\epsilon}{\rho}$. Note that in the equilibrium diffusion limit, $\epsilon = a T^4$ in the leading- and first-order which yields $\hat{e} = e^* = e + \frac{a T^4}{\rho}$.

Let assume that the Jacobian matrix associated to the change of variable proposed in \eqt{eq:chge_var} is denoted by $P$, i.e. $X=PY$, then the quadratic from in the new basis is derived as follows:
%
\begin{equation}
X^T A X =  (PY)^T A (PY) = Y^T (P^T A P) Y = Y^T \hat{A} Y \, ,
\end{equation}
%
where $\hat{A} = P^T A P$ is now the matrix of interest.

The first task consists in deriving the Jacobian matrix $P$:
%
\begin{equation}
P = \frac{\partial (\partial_x \rho, \partial_x e, \partial_x \epsilon)}{\partial (\partial_x \rho, \partial_x \hat{e}, \partial_x \epsilon)} = 
\begin{bmatrix}
\partial_{\partial_x \rho} \partial_x \rho & \partial_{\partial_x \hat{e}} \partial_x \rho & \partial_{\partial_x \epsilon} \partial_x \rho \\
\partial_{\partial_x \rho} e & \partial_{\partial_x \hat{e}} e & \partial_{\partial_x \epsilon} e \\
\partial_{\partial_x \rho} \partial_x \epsilon & \partial_{\partial_x \hat{e}} \partial_x \epsilon & \partial_{\partial_x \epsilon} \partial_x \epsilon
\end{bmatrix}
\end{equation}
%
From the definition of $\hat{e}$, it is derived that $\partial_x e = \partial_x \hat{e} - \frac{1}{\rho} \partial_x \epsilon + \frac{\epsilon}{\rho^2} \partial_x \rho$ which yields the following expression for the matrix $P$ and its transpose:
%
\begin{equation}\label{eq:jac}
P =
\begin{bmatrix}
1 & 0 & 0 \\
\frac{\epsilon}{\rho^2} & 1 & \frac{-1}{\rho} \\
0 & 0 & 1
\end{bmatrix}
\text{ and }
P^T =
\begin{bmatrix}
1 & \frac{\epsilon}{\rho^2} & 0 \\
0 & 1 & 0 \\
0 & \frac{-1}{\rho} & 1
\end{bmatrix}
\end{equation}
%
To simplify the derivation, the entropy functional form derived from the IGEOS is used:
%
\begin{equation}\label{eq:entropy}
s(\rho, e, \epsilon) = C_v \ln \left( \frac{e}{\rho^{\gamma-1}} \right) + \frac{4}{3} \frac{a^\frac{1}{4}}{\rho} \epsilon^\frac{3}{4} \, .
\end{equation}
%
Then, using the functional form of the entropy provided in \eqt{eq:entropy}, the matrix $A$ becomes:
%
\begin{equation}
A = 
\begin{bmatrix}
-\frac{C_v (\gamma-1)}{\rho^2} & 0 & 0 \\
0 & -\frac{C_v}{e^2} & 0 \\
0 & 0 & -\frac{a^\frac{1}{4}}{4\rho}\epsilon^\frac{-5}{4}
\end{bmatrix}
\end{equation}
%
We now compute $\hat{A}$ by first computing $Z=AP$ and then $\hat{A} = P^T Z = P^T A P$.
%
\begin{equation}
Z = 
\begin{bmatrix}
-\frac{C_v (\gamma-1)}{\rho^2} & 0 & 0 \\
-\frac{\epsilon C_v}{(\rho e)^2} & -\frac{C_v}{e^2} & \frac{C_v}{\rho e^2} \\
0 & 0 & -\frac{a^\frac{1}{4}}{4\rho}\epsilon^\frac{-5}{4}
\end{bmatrix} \, ,
\end{equation}
%
and 
%
\begin{equation}
\hat{A} = 
\begin{bmatrix}
-\frac{C_v}{\rho^2} \left((\gamma-1)+\left(\frac{\epsilon}{\rho e}\right)^2\right) & -\frac{\epsilon C_v}{(\rho e^2)} & \frac{C_v \epsilon}{\rho^3 e^2} \\
-\frac{\epsilon C_v}{(\rho e)^2} & -\frac{C_v}{e^2} & \frac{C_v}{\rho e^2} \\
\frac{\epsilon C_v}{\rho^3 e^2} & \frac{C_v}{\rho e^2} & -\frac{C_v}{(\rho e)^2}-\frac{a^\frac{1}{4}}{4\rho}\epsilon^\frac{-5}{4}
\end{bmatrix} \, .
\end{equation}
The matrix $\hat{A}$ is symmetric and needs to be compared against the matrix $A_{EDL}$ of the quadratic form obtained when considering the Equilibrium-Diffusion limit equations (leading-order continuity, momentum and total (material and radiation) energy equation and the second-order radiation energy density equation):
%
 \begin{equation}
 A _{EDL}= 
\begin{bmatrix}
\rho^{-2}\partial_{\rho} \left( \rho^2 \partial_{\rho} s \right) & \partial_{\rho,e^*} s & \partial_{\rho} \left( \rho \partial_{\epsilon} s \right) \\
 \partial_{\rho,e^*} s & \partial_{e^*,e^*} s & \partial_{e^*,\epsilon} s \\
 \partial_{\rho} \left( \rho \partial_{\epsilon} s \right) & \partial_{e^*,\epsilon} s & \partial_{\epsilon,\epsilon} s
\end{bmatrix}
\,.
\end{equation}
%
Note that to obtain the expression for $A_{EDL}$, the second-order radiation energy density is accounted for even if $\epsilon = aT^4$ at leading- and first-order.

Using the functional form of the entropy expressed as a function of $(\rho, e^*, \epsilon)$, 
%
\begin{equation}
s(\rho, \hat{e}, \epsilon) = C_v \ln \left( \frac{\hat{e}-\frac{\epsilon}{\rho}}{\rho^(\gamma-1)} \right) + \frac{4}{3} \frac{a^\frac{1}{4}}{\rho} \epsilon^\frac{3}{4} \, ,
\end{equation}
%
it can be shown that $A_{EDL} = \hat{A}$ when assuming $\hat{e} = e^*$. In the EDL, the radiation energy density equation is substituted by the simple relation $\epsilon = a T^4$ which is equivalent to removing the last row and last columns of the matrices $A_{EDL}$ and $\hat{A}$.
\end{document}