\documentstyle[10pt]{letter}

%%%%%% Letter Size Setup %%%%%%%%%%%%%%%%%%%%%%%%%%%%%%%%%%%%%%%%%%%%%%%%%%
%        \addtolength{\textwidth}{2.5cm}     %% For longer or shorter text width
        \addtolength{\topmargin}{-4.0cm}    %% For more or less top margin
       \addtolength{\textheight}{7.5cm}    %% For longer or shorter textheight
%        \addtolength{\oddsidemargin}{-1.25cm} %% For odd side margin (twoside)
                                            %% or margin (oneside)

\address{Jean Ragusa\\ 
Department of Nuclear Engineering \\
Texas A\&M University\\
College Station, TX 77843-3133, USA\\
phone: (979) 862 2033\\
e-mail: jean.ragusa@tamu.edu \vspace{0.5cm}}

%%%%%% The Signature  and Date %%%%%%%%%%%%%%%%%%%%%%%%%%%%%%%%%%%%%%%%%%%%

\signature{\vspace{-1.25cm}Marc Delchini, Jim Morel, Jean Ragusa}   


\begin{document}

\begin{letter}{Professor William Martin\\
    Editor,\\
    Journal of Computational Physics}
\date{\today}
%%%%%% More vertical space can be added here %%%%%%%%%%%%%%%%%%%%%%%%%%%%%%
%         \vspace{3.0cm}

\opening{Dear Professor Martin,}
         \vspace{0.25cm}
%%%%%% More vertical space can be added here %%%%%%%%%%%%%%%%%%%%%%%%%%%%%%

Please find attached a copy of our manuscript titled ``Numerical solution of the 1-D grey radiation hydrodynamics equations with an entropy-based artificial viscosity'' for submission to the {\em Journal of Computational Physics}. 

A viscous stabilization based on entropy production is derived for the $1$D grey radiation hydrodynamics equations. The technique is independent of the choice of spatial discretization and we have chosen to solve the equations with {\em continuous} finite elements. 
We extended the entropy-based stabilization, devised by Guermond et al. for Euler equations, to the radiation hydrodynamics equations (with grey non equilibrium diffusion).  Several typical 1-D radiation-hydrodynamic test cases with shocks (from Mach 1.05 to Mach 50) are computed to establish the ability of the technique at capturing and resolving shocks. Most of the numerical test cases are taken from Lowrie and Edwards (Radiative shock solutions with grey non equilibrium diffusion, {\it Shock Waves} (2008) 18:129-143).

 
This work follows closely prior works on Radiative shock solutions. The suggested reviewers are: Rob Lowrie (LANL), James Stone (Princeton University), and Robert Rieben (LLNL).
 

Thank you for considering this manuscript for publication in JCP.

%\vspace{-0.25cm}


%%%%%% More vertical space can be added here %%%%%%%%%%%%%%%%%%%%%%%%%%%%%%

%%%%%%% The Closing %%%%%%%%%%%%%%%%%%%%%%%%%%%%%%%%%%%%%%%%%%%%%%%%%%%%%%%
\closing{Best regards, }

\end{letter}
\end{document}

Rob Lowrie
lowrie@lanl.gov
Computational Physics Group (CCS-2)
Computer, Computational and Statistical Sciences Division Los Alamos National Laboratory Tel. (505) 667-2121

James Stone
Department of Astrophysical Sciences
Peyton Hall, Ivy Lane
Princeton, NJ 08544-1001
Office: 609-258-3815
e-mail: jstone (at) astro.princeton.edu 

Robert Rieben
Lawrence Livermore National Laboratory
Weapons and Complex Integration
Box 808, L-095 
7000 East Ave. L-095
Phone: (925) 422-3783
Fax: (925) 424-6163
Email: rieben1@llnl.gov
