%% Based on a TeXnicCenter-Template by Tino Weinkauf.
%%%%%%%%%%%%%%%%%%%%%%%%%%%%%%%%%%%%%%%%%%%%%%%%%%%%%%%%%%%%%
%%%%%%%%%%%%%%%%%%%%%%%%%%%%%%%%%%%%%%%%%%%%%%%%%%%%%%%%%%%%%
%% HEADER
%%%%%%%%%%%%%%%%%%%%%%%%%%%%%%%%%%%%%%%%%%%%%%%%%%%%%%%%%%%%%
\documentclass{article}

%% Packages for Graphics & Figures %%%%%%%%%%%%%%%%%%%%%%%%%%
\usepackage{graphicx} %%For loading graphic files
%% Math Packages %%%%%%%%%%%%%%%%%%%%%%%%%%%%%%%%%%%%%%%%%%%%
\usepackage{amsmath}
\usepackage{amsthm}
\usepackage{amsfonts}
\usepackage{color}
\usepackage{listings}
\lstset{language=C++}
\usepackage{lscape} 
\usepackage{float}
\usepackage{graphicx}
\usepackage{caption}
\usepackage{subcaption}
%\usepackage{landscape}
\usepackage{xspace}
\usepackage{color}
\textwidth = 450pt
\def\fxnote#1{\marginpar{\textcolor{green}{#1}}}
\def\fxwarning#1{\marginpar{\textcolor{red}{#1}}}
\usepackage[a4paper]{geometry}
% common reference commands
\newcommand{\eqt}[1]{Eq.~(\ref{#1})}                     % equation
\newcommand{\fig}[1]{Fig.~\ref{#1}}                      % figure
\newcommand{\tbl}[1]{Table~\ref{#1}}                     % table
\newcommand{\sect}[1]{Section~\ref{#1}}                     % section
\newcommand{\subsect}[1]{Subsection~\ref{#1}}                     % subsection
\newcommand{\app}[1]{Appendix~\ref{#1}}                     % appendix

\newcommand{\ie}{i.e.,\@\xspace}
\newcommand{\eg}{e.g.,\@\xspace}
\newcommand{\psc}[1]{{\sc {#1}}}
\newcommand{\rs}{\psc{R7}\xspace}

\title{Entropy minimum principle for the 1-D Grey non-equilibrium radiation-hydrodynamic equations.}
\author{Marc-Olivier Delchini}

%%%%%%%%%%%%%%%%%%%%%%%%%%%%%%%%%%%%%%%%%%%%%%%%%%%%%%%%%%%%%
%% DOCUMENT
%%%%%%%%%%%%%%%%%%%%%%%%%%%%%%%%%%%%%%%%%%%%%%%%%%%%%%%%%%%%%
\begin{document}
\maketitle
\noindent
\textbf{Objective:} demonstrate that the entropy minimum principle holds for the Grey non-equilibrium radiation-hydrodynamic equations with source terms when using the definition of the entropy derived from the study of the hyperbolic terms of the radiation-hydrodynamic equations.
\\
\\
The $1$-D system of equations is: 
%
\begin{equation}
\label{eq:radhydro}
\left\{
\begin{array}{llll}
\partial_t \rho + \partial_x \left( \rho u \right) = 0\\
\partial_t \left( \rho u \right) + \partial_x \left( \rho u^2 + P \right) = \frac{1}{3} \partial_x \epsilon \\
\partial_t \left( \rho E \right) + \partial_x \left[ u \left(\rho E + P \right) \right] = - \sigma_a c \left( a T^4 - \epsilon \right) - \frac{u}{3} \partial_x \epsilon \\
\partial_t \epsilon + \frac{4}{3} \partial_x \left( u \epsilon  \right) = \partial_x \left( D \partial_x \epsilon \right) + \sigma_a c \left( a T^4 - \epsilon \right) + \frac{u}{3} \partial_x \epsilon 
\end{array}
\right.
\end{equation}
All variables have the same definition as in the paper (I use the same notations for consistency). The entropy equation is derived from the above equations and has the following form:
%
\begin{align}
\rho \frac{Ds}{Dt} = \underbrace{\rho s_\epsilon \partial_x \left( D \partial_x \epsilon \right)}_{R_1} + \underbrace{\left( \rho s_\epsilon - s_e  \right) \sigma_a c \left( a T^4 - \epsilon \right)}_{R_2} \nonumber
\end{align}
%
where $s(\rho, e, \epsilon) = s_{Euler} (\rho,e) + \frac{\rho^{(0)}}{\rho}\tilde{s}(\epsilon)$ from Appendix A. We also showed that $s_e = T^{-1} \geq 0$ and $\tilde{s}(\epsilon) = \frac{4 a^\frac{1}{4}}{3} \epsilon^\frac{3}{4}$. In the remaining of the derivation, $\rho^{(0)}$ is assumed to be equal to 1.\\
We now investigate the sign of each of the residuals namely $R_1$ and $R_2$.
%
\begin{itemize}
%
\item The first residual $R_1$ can be recast by noticing that $s_\epsilon = \frac{1}{\rho} \frac{d\tilde{s}(\epsilon)}{d \epsilon}$ which yields:
%
\begin{equation}
R_1 = \frac{d \tilde{s}}{d \epsilon} \partial_x \left( D \partial_x \epsilon \right) = \tilde{s}' \partial_x \left( D \partial_x \epsilon \right) 
\end{equation}
%
Integrating by part, we obtain:
%
\begin{align}\label{eq:R1}
R_1 &= \partial_x \left( \tilde{s}' D \partial_x \epsilon \right) - D \partial_x \epsilon \ \partial_x \tilde{s}'  \nonumber \\
R_1 &= \partial_x \left( D \partial_x \tilde{s} \right) - D \tilde{s}'' \left( \partial_x \epsilon \right)^2
\end{align}
%
The second term in \eqt{eq:R1} is positive since $\tilde{s}$ is concave by definition, i.e. $\tilde{s}'' \leq 0$ (this is showed in Appendix A of the paper). The first term in \eqt{eq:R1} is a conservative term and thus will only yields boundary terms when integrating over time and space.
%
\item Using the definitions of $\tilde{s}$ and $s_e$, $R_2$ can be rewritten as:
%
\begin{align}\label{eq:R2}
R_2 &= \left( \left( \frac{a}{\epsilon}\right)^\frac{1}{4} - \frac{1}{T} \right) \sigma_a c \left( aT^4 - \epsilon \right) \nonumber \\
R_2 &= \frac{\sigma_a c}{T \epsilon} \left( \left(\frac{aT^4}{\epsilon}\right)^\frac{1}{4} - 1 \right) \left( \frac{a T^4}{\epsilon} -1  \right) \nonumber \\
R_2 &= \frac{\sigma_a c}{T \epsilon} \mathbb{D} \left( \frac{a T^4}{\epsilon} -1  \right) \, ,
\end{align}
%
with $\mathbb{D} = \left(\frac{aT^4}{\epsilon}\right)^\frac{1}{4} - 1$ (same definition as in the paper).
Assuming that $\epsilon \geq 0$ and using the short notation $\mathbf{d} = \frac{a T^4}{\epsilon}$, \eqt{eq:R2} can be recast as follows:
%
\begin{align}\label{eq:R2_2}
R_2 &= \frac{\sigma_a c}{T \epsilon} \mathbb{D} \left( \mathbf{d}^\frac{1}{2}-1 \right) \left( \mathbf{d}^\frac{1}{2} + 1 \right) \nonumber \\
R_2 &= \frac{\sigma_a c}{T \epsilon} \mathbb{D} \left( \mathbf{d}^\frac{1}{4}-1 \right) \left( \mathbf{d}^\frac{1}{4} + 1 \right)  \left( \mathbf{d}^\frac{1}{2} + 1 \right) \nonumber \\
R_2 &= \frac{\sigma_a c}{T \epsilon} \mathbb{D}^2\left( \mathbf{d}^\frac{1}{4} + 1 \right)  \left( \mathbf{d}^\frac{1}{2} + 1 \right) \geq 0 \, ,
\end{align}
%
since $\mathbb{D} = \mathbf{d}^\frac{1}{4} + 1$.
\end{itemize}
%
From \eqt{eq:R1} and \eqt{eq:R2_2}, we devise that:
%
\begin{equation}
\rho \frac{Ds}{Dt} \geq 0 \nonumber
\end{equation}
%
\end{document}
